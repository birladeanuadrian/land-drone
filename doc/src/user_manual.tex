

\chapter{User's manual}
\label{ch:user-manual}

\section{Deployment}
\label{sec:user-manual-deployment}

In order to run this project, someone needs to setup the following:
\begin{enumerate}
    \item A Drone with a Raspberry PI connected to the internet
    \item A Cloud server to run proxy and image processor
    \item A browser
\end{enumerate}

\subsection{Robot}
\label{subsec:user-manual-robot}
There are multiple ways to build a robot from a hardware perspective, \
so I will focus on the software.
The robot application requires the following to be installed:
\begin{itemize}
    \item python 3.8 (latest version)
    \item nodejs (version 14 or later)
    \item a rabbitmq server (with optional management interface)
    \item a no-ip dns client and a name for remote access
\end{itemize}
Once all dependencies have been installed, the components should be started \
manually from an ssh client.
The no-ip free dns rezervation is only available for 30 days, after which \
it should be renewed from their web application.
Once the DNS client is running on the robot, you should open port 22 for \
the DNS from the no-ip web interface.

The rabbitmq server should start automaticlly on boot.
If it did not start, it can started by running the command \
$sudo service rabbitmq-server start$.
You can check if rabbitmq is running by opening a browser and navigating \
to $http://<your-robot-dns>:15672$ or by running the command
$rabbitmqctl status$ on the robot.

To run the engine control on the drone, you must first install its dependencies.
This can be done by running the command $pip3 install -r requirements.txt$ on \
the robot in the folder which contains the python source files.
After that it can be run by running the command $python3 main.py$.

As for the outer comms, after deploying the source files you must also \
install the dependencies by running $npm install$ in the folder where you \
deployed the files.
Then you must compile the typescript files into javascript by running \
$npm run compile$.
Finally you can run it with the command $node dist/producer1.js$.


\subsection{Cloud}
\label{subsec:user-manual-cloud}
The robot was tested using a server deployed on the Google Cloud,
but any cloud provider can be used.
The cloud components's requirements are at least 2 CPUs, 10GB RAM and
a GPU.
There are 2 ways to configure a virtual machine:
\begin{enumerate}
    \item Manually deploy and configure a virtual machine (the virtual \
        machine needs to be configured according to the instructions in the \
        Dockerfile)
    \item Deploy a new instance from the already configured snapshot
        (only works in Google Cloud, where the snapshot was created)
\end{enumerate}
Once the virtual machine has been fully configured, one needs to ssh into \
it and manually start the components.
The cloud proxy can be started with the command $npm start$, while the \
cloud image processor can be started with the command $python3 main.py$.
All warnings about \textit{NUMA} should be ignored.

If you installed the virtual machine manually, then you must also \
install dependencies and compile typescript for the proxy and image \
processor, similarly to how it was done on the robot.

\subsection{Browser}
\label{subsec:user-manual-browser}
A browser is required to view the video feed and control the drone.\
It should preferably be on a desktop, since it offers a larger \
viewport.

\section{Operation}
\label{sec:user-manual-operation}
As soon as it starts, the robot should connect to the proxy server and \
send images to the image processor, without any user input.
In the browser, the image should appear soon after the robot starts.
Before issuing any commands, one needs to first enter the controller \
secret and login with it.
Otherwise the control commands will not work.

Below the video there is a table with the state of the robot: acceleration \
and direction.
In order to change acceleration, one should press one the \textit{-} or \
\textit{+} buttons.
As for direction, it can be changed by clicking on one of the two arrows.
To start a recording, click on the button \textit{Start recording}, and to \
stop it press on the button \textit{Stop Recording}.
The stop recording button will immediately start a browser download of the \
video.

In order to detect people, click on the button \textit{Detect People}, \
and to stop detection click on the button \textit{Stop detection}.

\section{Known Issues}
\label{sec:manual-known-issues}
\begin{itemize}
    \item if the FPS rate set in the robot iss too great, the image might \
        freeze; you should restart the robot's outer comms with a lower \
        FPS rate
    \item browser doesn't auto-login controller; if you logged in \
        to control robot and then refresh the tab, you should login \
        again
    \item when the proxy server exits, all components lose their login \
        session (cloud image processor and web app); to recover login sessions, \
        the user in the web app should login again and the cloud image \
        processor should be restarted
\end{itemize}