\usepackage{graphicx}


\chapter{Project Objectives and Specifications}
\label{ch:specification}

\section{Problem specification}
\label{sec:specification-specification}

Internet-controlled drones can take several forms and serve multiple purposes. \
This specific drone is a land drone (car) that streams video-only, not audio for now, and accepts \
commands from a web interface. \
Because 5G networks aren't accessible except via high-end smartphones, the robot will be connected \
to the internet via a 4G LTE mobile modem.

I have chosen a land drone because it is both easier and more affordable to build and test. \
However, most components (cloud server, web interface, drone - server communication) can be reused \
in order to build other types of drones (naval and flying drones).

Additionally, the drone will have person detection and face recognition capabilities in order to \
meet certain security-related use cases, such as monitoring remote locations.

The current state-of-the-art in remote controlled drones is met on military drones. \
According to .. % todo: quote: https://www.forbes.com/sites/sebastienroblin/2019/09/30/dont-just-call-them-drones-a-laypersons-guide-to-military-unmanned-systems-on-air-land-and-sea/#76c957e62b00
the most popular control mechanisms are by radio and by satellite uplink. \
While the radio control requires a certain proximity of the drone to the operator (i.e., 100 miles), \
drones controlled via satellite uplink introduce a certain latency (image transmitted from a drone \
in Afghanistan may take up to 1.2 seconds to reach the operator in the US). \
High-Speed internet might provide a third control mechanism, that could remove the range limitation \
of the radio and decrease the latency of current satellite uplinks.


\section{General Objectives}
\label{sec:specification-objectives}
The primary objective of this project is to create a drone that can be remotely controlled from \
half-way around the world.

\subsection{Use Cases}
\label{subsec:use-cases}
The project will need to meet the following use cases:

\subsubsection{Login}
\textbf{Primary Actor:} Drone Operator
\textbf{Stakeholders and Interests:}
\begin{enumerate}
    \item \textit{Drone Operator}: wants to be the sole person who controls \
            the drone at a given moment
    \item \textit{Drone owner}: wants to limit access to the drone video feed \
            and control to authorized personnel only
\end{enumerate}
\textbf{Postcondition:} The operator is granted access to the land drone
\textbf{Main Success Scenario}
\begin{enumerate}
    \item the operator goes to the drone url
    \item the operator inserts the username and password
    \item the cloud server validates the username and password
    \item the cloud server validates that no one else controls the drone
    \item the cloud server gives the operator access to the drone video \
            feed and controls
\end{enumerate}
\textbf{Extensions}
\begin{itemize}
    \item 3. The username or password is incorrect
        \begin{itemize}
            \item an error message is shown
            \item video feed and controls are disabled until the operator \
                    reloads the page
        \end{itemize}
    \item 4. Another operator already controls the drone
        \begin{itemize}
            \item the current operator is given access only to the video \
                    feed, while the first operator retains access to the controls
        \end{itemize}
\end{itemize}

\subsubsection{Video Feed}




As stated above, the primary objective of \applicationTitle{} is to create a menu generator application that can solve \
most of the problems of the other existing alternatives. \
This will allow older people to maintain a healthy diet by offering them recommendations for meals for the entire day.

In order to accomplish its purpose, the application needs to gather a certain amount of information about \
each user. \
This includes personal information (age, gender, weight, height, physical activity level). \
In addition \
to this the system also requires information about specific allergies (in order to exclude certain foods when generating \
a menu) and about current chronic diseases (so that a nutritionist will be aware of them when validating the user \
profile). \
Using the information above, a profile is generated that include the body mass index (used to determine if a \
person is normal weight, underweight, overweight or obese) and recommended nutrient, caloric, vitamin and mineral intake \
for a day. \
The recommended daily nutritional intake levels represent the amount of nutrients that are sufficient in order \
to meet the requirements of 97\-98\% of healthy people. \
However, no menu can be generated until the profile is reviewed \
by a nutritionist.

The information presented above will be the input to a choice-based algorithm that will generated an optimal recommendation for a \
menu for a day. \
The other inputs will include certain non-functional requirements (preferred price, delivery time, geographical \
location of vendor). \
Since one of the objectives of this system is to generated meals that can be delivered to the users' homes, \
the system will require extensive information about local sellers. \
This will include, along information about geographical \
location, information about the meals they offer(type, price, cooking time, ingredients, nutritional values, weight). \
This will require food vendors to have dedicated accounts in the system in order to edit their company's offers. \
Using all \
this information, an algorithm will compute the best possible menu to satisfy an elder's nutritional \
daily intake requirements and non-functional requirements. \
A use case diagram can be seen in figure 2.1 \ref{fig:usecase1}.
A basic activity flow can be seen in diagram 2.2.

\begin{figure}[ht]
    \label{fig:usecase1}
    %\centering
    \includegraphics[width=15cm, height=50cm,keepaspectratio]{staruml/UseCaseDiagram1.png}
    \caption{Use Case Diagram}
\end{figure}

\begin{figure}[ht]
    \label{fig:activity1}
    %\centering
    \includegraphics[width=15cm, height=50cm,keepaspectratio]{staruml/ActivityDiagram1.png}
    \caption{Use Case Diagram}
\end{figure}

However, all the information above leads to a very large search space, possibly leading to performance issues. \
This makes an exhaustive search very time consuming, since there are very many possible combinations. \
The users desire a fast way to \
generate menus, especially if they don't like one of the menus generated and want to generate another one. \
This is why I have decided to use a choice-function based hyper heuristic\
\footnote{A high level hyper heuristic that applies other, low level heuristics on a search space}.


\section{Functional Requirements}
\label{sec:specification-functional}

In software engineering, a functional requirement is a function of a system or its components. \
Such a function consists of a specific set of inputs, a behavior and an output. \
Functional requirements describe what the system \
is expected to do in order to accomplish its purpose and reach the desired results.

The functional requirements I have set out to achieve are mosyly based on the general objectives presented above.\
First, the application requires extensive information about various food vendors and their offers, alongside some \
personal information about the elder users. \
Once the user data is in the database, the application should be able \
to generate user profiles describing the recommended daily nutritional values.

Ideally, there would be 3 types of users: elders, nutritionists and food distributors, each having to log in \
into the application. \
An elder should be able to edit his contact information, his personal health information, choose his nutritionist, \
generate a menu for \
a day and view the menu's properties (nutritional values). \
A nutritionist should be able to view the elders who \
have chosen him as their nutritionist, view the elder's personal information, edit the recommended nutritional \
intake values and validate a certain user profile. \
An elder cannot create a menu until his profile has been validated by a nutritionist. \
Food distributors should be able to add, edit and delete meal packages.

\section{Non-Functional requirements}
\label{sec:specification-non-functional}
In software engineering, a non-functional requirement refers to a system's quality characteristics and attributes, \
unlike functional requirements which refer to a system's functions. \
Non-functional requirements are used to \
evaluate all operations of the system, not just a specific component or behavior. \
They can be split into 2 \
categories: execution requirements (like security, performance) and evolution requirements (like scalability,
testability, extensibility). \
In the following subsections we will present in detail all the non-functional requirements.

\subsection{Security}
\label{subsec:specification-security}
%todo: fix man-in-the-middle
The system is a Java web application, therefore using HTTP communication. \
In order to prevent any \
man-in-the-middle-attack, we can setup HTTP encryption either at Tomcat level, or setup a \
reverse proxy using a second web server (Apache, Nginx) and set up encryption at the second web \
server.

Since the application deals with highly sensitive personal information (allergies, diseases), we \
need to set up authentication and authorization. \
Users need to login before they can interact further with the system. \
We must also make sure that users can only see what they are supposed to see. \
For example, we must make sure that food distributors cannot see user profiles. \
Thus, we must create authorization levels for each type of user.

In order to secure user data from the database in case of a breach, we will store user passwords \
not in plain text, but as SHA-256 hash.

\subsection{Performance}
\label{subsec:specification-performance}
Performance is one of the most important requirements of the system and is measured in several \
possible ways:
\begin{itemize}
    \item Response time
    \item Time to generate a menu
    \item Quality of the generated menu
\end{itemize}

The web response time is a general non-functional requirement and refers to how long the user \
must wait for a page to load and be fully functional. \
In order to attract users, this time should be as low as possible. \
Otherwise, users may decide to use other web applications that load faster.

The time to generate a menu should also be short, no matter the search space. \
This was one of \
the primary requirements when designing and implementing the search hyper heuristic. \
The quality of the generated menu is also important in order to evaluate the algorithm.

\subsection{Scalability and extendibility}
\label{subsec:specification-scalability}
Scalability of the system refers to the ease with each the system can accommodate an increased \
number of users and data from distributors without changing the core of the application. \
Since \
this is a web application, it can easily be scaled horizontally by adding more servers with a \
high availability proxy in front.

Extendibility refers to the ease with which new functional requirements can be accommodate as \
the number of users and their expectations increase.

\subsection{Usability}
\label{subsec:specification-usability}
One of the primary facts that has to be taken info consideration when designing and \
implementing the application was that the main users will be elders, most of whom are \
not very accustomed to technology. \
This is why the application, and especially the side \
that elders interact with, needs to be very intuitive and easy to understand.