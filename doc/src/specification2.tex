\usepackage{graphicx}


\chapter{Project Objectives and Specifications}
\label{ch:specification}

\section{Problem specification}
\label{sec:specification-specification}

Internet-controlled drones can take several forms and serve multiple purposes. \
This specific drone is a land drone (car) that streams video-only, not audio for now, and accepts \
commands from a web interface. \
Because 5G networks aren't accessible except via high-end smartphones, the robot will be connected \
to the internet via a 4G LTE mobile modem.

I have chosen a land drone because it is both easier and more affordable to build and test. \
However, most components (cloud server, web interface, drone - server communication) can be reused \
in order to build other types of drones (naval and flying drones).

Additionally, the drone will have person detection and face recognition capabilities in order to \
meet certain security-related use cases, such as monitoring remote locations.

The current state-of-the-art in remote controlled drones is met on military drones. \
According to .. % todo: quote: https://www.forbes.com/sites/sebastienroblin/2019/09/30/dont-just-call-them-drones-a-laypersons-guide-to-military-unmanned-systems-on-air-land-and-sea/#76c957e62b00
the most popular control mechanisms are by radio and by satellite uplink. \
While the radio control requires a certain proximity of the drone to the operator (i.e., 100 miles), \
drones controlled via satellite uplink introduce a certain latency (image transmitted from a drone \
in Afghanistan may take up to 1.2 seconds to reach the operator in the US). \
High-Speed internet might provide a third control mechanism, that could remove the range limitation \
of the radio and decrease the latency of current satellite uplinks.

% todo: return here at home
Possible commercial use cases for this drone include wildlife observation,


\section{General Objectives}
\label{sec:specification-objectives}
The primary objective of this project is to create a drone that can be remotely controlled from \
half-way around the world.

\subsection{Use Cases}
\label{subsec:use-cases}
The project will need to meet the following use cases:

\subsubsection{\textbf{Login}}

\textbf{Primary Actor:} Drone Operator

\textbf{Stakeholders and Interests:}
\begin{enumerate}
    \item \textit{Drone Operator}: wants to be the sole person who controls \
            the drone at a given moment
    \item \textit{Drone owner}: wants to limit access to the drone video feed \
            and control to authorized personnel only
\end{enumerate}

\textbf{Postcondition:} The operator is granted access to the land drone

\textbf{Main Success Scenario}
\begin{enumerate}
    \item the operator goes to the drone url
    \item the operator inserts the username and password
    \item the cloud server validates the username and password
    \item the cloud server validates that no one else controls the drone
    \item the cloud server gives the operator access to the drone video \
            feed and controls
\end{enumerate}

\textbf{Extensions}
\begin{itemize}
    \item 3. The username or password is incorrect
        \begin{itemize}
            \item an error message is shown
            \item video feed and controls are disabled until the operator \
                    reloads the page
        \end{itemize}
    \item 4. Another operator already controls the drone
        \begin{itemize}
            \item the current operator is given access only to the video \
                    feed, while the first operator retains access to the controls
        \end{itemize}
\end{itemize}

\subsubsection{\textbf{View Drone Footage}}
\textbf{Primary Actor:} Drone Operator

\textbf{Stakeholders and Interests:}
\begin{enumerate}
    \item \textit{Drone Operator:} wants to view a live video feed from the drone
    \item \textit{Drone Operator and Owner} want to view the drone footage at a later date
\end{enumerate}

\textbf{Preconditions:} Drone operator has logged in and is the sole operator for the drone

\textbf{Main Success Scenario}
\begin{enumerate}
    \item Drone operator goes to the drone page
    \item Drone operator clicks on the \textbf{Start video feed} button
    \item The drone starts transmitting video feed
    \item The frontend shows the live video feed
\end{enumerate}

\textbf{Extensions}
\begin{itemize}
    \item The drone operator wants to save the video feed so that he can watch it at a later date
        \begin{itemize}
            \item After clicking on the \textbf{Start video feed} button, the operator clicks on the \
                    \textbf{Record feed} button
            \item The system starts recording the video feed
            \item The operator clicks on the \textbf{Stop recording} button
            \item The operator click on the \textbf{Download recording button}
            \item The system downloads the recording
        \end{itemize}
    \item The drone operator wants people to be highlighted
            \begin{itemize}
                \item After clicking on the \textbf{Start video feed}, but before clicking on the \
                        \textbf{Record feed} button, the operator presses on the \textbf{Highlight people} button
                \item The system shows the video feed, highlighting people and writing the names of the
                        known ones
            \end{itemize}
    \item A second operator wants to view the real time footage
            \begin{itemize}
                \item The second operator logs in
                \item The second operator views the same footage as the first
                \item The system disables all actions for the second operator (start recording, highlight people)
            \end{itemize}
\end{itemize}

\subsubsection{Control the drone}

\textbf{Primary Actor:} Drone Operator

\textbf{Stakeholders and Interests:}
\begin{enumerate}
    \item \textit{Drone operator} wants to move the drone in specific directions in order to meet his objectives
\end{enumerate}

\textbf{Preconditions:} Drone operator has logged in

\textbf{Main Success Scenario}
\begin{enumerate}
    \item The drone operator starts video feed and waits for it
    \item The drone operator controls the drone using the keyboard arrow keys
\end{enumerate}

\textbf{Extensions}
\begin{itemize}
    \item A second operator enters the drone page while the first operator control the drone
            \begin{itemize}
                \item The system disables the controls for the second operator
            \end{itemize}
\end{itemize}

% todo: go here
The presented use cases can be applies in the following fields:
\begin{enumerate}
    \item \textbf{Wildlife observation} the robot can be used to observe wild animals from a distance in areas with \
        decent internet coverage (minimum 10Mbps upload speed) % todo: maybe update value
    \item \textbf{Perimeter patrol} the robot can be used to patrol remote open areas where it's not possible to \
        mount security cameras (forests, borders)
\end{enumerate}


\section{Functional Requirements}
\label{sec:functional-requirements}

In software engineering, a functional requirement is a function of a system or its components. \
Such a function consists of a specific set of inputs, a behavior and an output. \
Functional requirements describe what the system \
is expected to do in order to accomplish its purpose and reach the desired results.

The functional requirements I have set out to achieve are mostly based on the use cases presented above. \
They are the following:
\begin{enumerate}
    \item live video feed from the drone to a web interface
    \item full control of the drone from a web interface
    \item recording capabilities for the live video feed
    \item option to highlight people and show their names
    \item limit of one active operator, with multiple passive operators having access to just the video feed
\end{enumerate}

\section{Non-Functional Requirements}
\label{sec:non-functional-requirements}

In software engineering, a non-functional requirement refers to a system's quality characteristics and attributes, \
unlike functional requirements which refer to a system's functions. \
Non-functional requirements are used to \
evaluate all operations of the system, not just a specific component or behavior. \
They can be split into 2 \
categories: execution requirements (like security, performance) and evolution requirements (like scalability,
testability, extensibility). \
In the following subsections we will present in detail all the non-functional requirements.

\subsection{Performance}
\label{subsec:specification-performance}
Performance is one of the most important requirements of the system and is measured in several possible ways:
\begin{itemize}
    \item Video latency
    \item Video FPS
    \item Response time to commands
\end{itemize}

The video latency is measured in milliseconds since the image displayed was taken. \
Since the project won't be using 5G internet networks, but 4G LTE, I set out to achieve an average of \
15FPS with a latency of 500ms and a command response time of 100ms.

\subsection{Deployment}
\label{subsec:deployment}
The deployment will consist of hardware deployment (the drone itself plus its configuration) and software \
deployment (the control UI and other proxy and processing servers). \
The first step is the software deployment, which will be done in a kubernetes cluster. \
This can be done with relative ease. \
THe second step is the drone deployment, which includes a step for configuring the drone (writing the \
ip address to which video feed will be sent and the ip address from which commands will be received).

\subsection{Scalability and extendability}
\label{subsec:specification-scalability}
Scalability of the system refers to the ease with each the system can accommodate an increased \
number of users and data from distributors without changing the core of the application. \
On the one hand, the  application is centered on a single drone, with several possible operators. \
On the other hand, the application will be deployed in a Kubernetes cluster, meaning a new instance can easily \
be deployed for more drones.

Extendability refers to the ease with which new functional requirements can be accommodate as \
the number of users and their expectations increase.  % todo: update extendability

\subsection{Usability}
\label{subsec:specification-usability}
% todo: complete usability

\subsection{Security}
\label{subsec:specification-security}
% todo: complete security
