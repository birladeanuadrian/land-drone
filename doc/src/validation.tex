
\chapter{Testing and Validation}
\label{ch:validation}

\section{Hardware and Robot Testing}
\label{sec:validation-hardware}
During robot construction, the first step was to test each DC motor \
individually.
I connected each robot directly to batteries to make sure that \
it works.
After this step, the next testing step was when the drone was fully \
assembled, and I needed to validate that it could go forward and backward.
I placed the chassis on a small box so that the wheels remained in air \
and the robot couldn't move,
It was at this point that I discovered that some motors were placed in a \
reverse direction and I needed to reverse their acceleration so that \
they would go in the same direction as the other 2 motors.

While developing the cloud components that interacted directly with the \
robot, I connected the robot to the local Wi-Fi network.
Only after having finished the robot and the cloud components did I \
connect it to 4G.

\section{Cloud and Web}
\label{sec:validation-individual}

The first software component to be implemented was the udp \
packer-unpacker.
I tested it by deconstructing and then reconstructing an image, \
and after several optimizations, I managed to pack an image in nodejs \
in 4 ms.

The next major component was the image transmission.
Its testing and development was done in 3 phases.
During development and initial testing I used my laptop's camera \
instead of the robot's with locally run image processor and proxy server.
The testing procedure was to visually check the image and the transmission delay.
A successful local test meant a transmission delay of less than 100ms with \
an approximate 10 FPS.
After I successfully tested locally, I deployed the proxy server and \
the image processor in the cloud on a GCP virtual machine and reran the \
tests.
Finally, I deployed the image generator on the robot while connected to \
the 4G and reran the tests.

The object detection algorithm was tested separately on a local video.
Several different algorithms were tested as well, but MobileNet had \
the best speed/performance ratio.

While testing the cloud image processor, I discovered that if the \
frequency at which the drone sens images is \
too large, a hidden occurs that causes the processor to freeze and \
to stop processing any other images correctly, and as a result the \
video in the browser freezes.
My best guess is that the processor cannot read all udp packets in time, \
leading udp packets to gather in an OS buffer.
Shortly, the number of udp packets in the buffer becomes too high \
and packets are dropped, which leads to the cloud processor becoming \
unable of reconstructing images.
However, more testing/deubgging is required in order to trace \
the exact issue.

\section{Final Measurements}
\label{sec:final-measurements}

At the end I took some final measurements checking the performance of \
image and command transmission.
I did the measurements both with and without TLS, to measure the effect \
of encryption on performance, which I found to be negligible.
For the tests, I took into consideration if object detection was enabled, \
cloud processing times, total delay, FPS and the number of people in the \
image (it could affect the performance of the object detection \
algorithm).
The results can be seen in table ~\ref{tab:final-results}

\begin{table}[ht]
    \caption{Final Performance Results}
    \centering
    \begin{tabular}{|c|c|c|c|c|c|}
        \hline\hline
        TLS & Detection & People & Image Processing Time & Total Delay & FPS \\
        \hline
        no  & yes & 1   & 60 & 133  & 9 \\
        no  & yes & 2   & 64 & 137  & 9 \\
        no  & yes & 0   & 59 & 132  & 9 \\
        no  & no  & \_  & 0  & 65   & 10 \\
        yes & no  & \_  & 0  & 67   & 11 \\
        yes & yes & 1   & 58 & 128  & 10 \\
        yes & yes & 0   & 55 & 123  & 10 \\
        yes & yes & 2   & 58 & 127  & 10 \\
        \hline
    \end{tabular}
    \label{tab:final-results}
\end{table}

As can be observed from the data, TLS doesn't seem to bring overhead.
The smaller times for TLS could be caused by the iamge processor taking \
less time to process the image.
Additionally, the number of people in the image doesn't impact the \
performance of the object detector.

Since in the specifications I wrote that the aim of this project was \
to design a system that would allow people to control a drone from \
anywhere they are with any type of device, I decided to test the \
feed video performance on a smartphone.
I ran the tests on a Samsung S20 connected to the local wi-fi \
network.
The smartphone reported a total delay of 1.8 seconds, but when I \
compared it to the video on the laptop I saw that they were \
synchronized, with a delay of 70ms.
The problem probably came from the fact that the laptop time was \
synchronized with windows servers, while the smartphone \
was sycnhronized with Samsung servers.
If there was indeed an additional delay on the smartphone, it \
wasn\'t one that could be observed with the eye.

I decided to repeat the test with a Samsung A6+.
This time, the smartphone reported a delay of -350ms,
but when compared to the image on the laptop, it was \
also synchronized.
Though the results didn't directly show it, the tests were \
a success.
