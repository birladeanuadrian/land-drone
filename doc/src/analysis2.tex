
\chapter{Analysis and Theoretical Foundation}
\label{ch:analysis}

% todo: system sequence diagram
% todo: aici: schema globala, schema hardware ()\robot + sensoes + gprs + cloud, 2 scheme software
% asmchart
%

%Together with the next chapter takes about 60\% of the whole paper
%
%The purpose of this chapter is to explain the operating principles of the implemented application.
%Here you write about your solution from a theory standpoint - i.e. you explain it and you demonstrate its theoretical properties/value, e.g.:
%\begin{itemize}
% \item used or proposed algorithms
% \item used protocols
% \item abstract models
% \item logic explanations/arguments concerning the chosen solution
% \item logic and functional structure of the application, etc.
%\end{itemize}
%
%{\color{red} YOU DO NOT write about implementation.
%
%YOU DO NOT copy/paste info on technologies from various sources and others alike, which do not pertain to your project.
%}

%\section{Title}
%\section{Other title}
\section{Algorithms}
\label{sec:analysis-algorithms}
 The project uses several algorithms from several different computer \
vision fields: object detection, object tracking and person recognition.

\subsection{Object detection}
\label{subsec:analysis-object-detection}
For object detection I have chosen to use deep neural networks due to their \
increased performance (speed and accuracy) compared to other ML-based algorithms.\
More specifically, I decided to use the  MobileNetV2~\cite{mobilenet2} neural \
network with SSD (single shot detection). \
MobileNet works as a feature extractor, with the output of its last layers \
connected to the input of the SSD network.

% todo: move this to another chapter
In order for deep neural networks to be effective, they need to be trained on \
large datasets. \
Some common such datasets are:
\begin{itemize}
    \item the Microsoft COCO dataset ~\cite{COCO} (about 300k images, out of \
            which more than 200k are labeled)
    \item the KITTI dataset ~\cite{kitti}
    \item Open Images dataset ~\cite{openimages} (over 1.7 million images)
\end{itemize}

A neural network takes a considerable amount of time to train on such datasets \
even with state-of-the-art hardware. \
For this reason, there are a number of pre-trained models available on the \
internet.


\section{Robot control}
\label{sec:analysis-robot-control}
The software components running on the robot are the engine control and the \
I/O control (handling all outside communication and the camera itself for \
performance reasons). \
In order to enforce low couping between these components and other components \
that could be added in the future, they are deployed separately and run in \
different processes. \
The only means of communication is a queue, more specifically a RabbitMQ that \
runs on the robot. \
The queue is used in order to transmit commands from the I/O control to the \
engine control for now.


\subsection{Image transmission}
\label{subsec:analysis-image-transmission}
 I needed to devise an algorithm that would allow to one source to send images to several receivers that are
 not on the same network and at considerable distances, and at the same time having a delay as small as possible.
 Since the sender and the receivers won't be on the same network, they will need to interact via a 3rd component,
 a server that runs in cloud.

 The image needs to go through 2 different paths in order to get from the robot to the \
user that controls it. \
The first path is from robot to the cloud server that acts as a proxy. \
Since this communication is initiated by the client, the UDP protocol, which is faster than \
TCP, can be used. \
However, UDP packets have a small upper size limit (~500 bytes), which if surpassed, may cause \
the packet to be dropped on the way. \
For this reason, each image needs to be split into several smaller packets that can be safely \
transmitted over UDP. \
Since the protocol does not guarantee the order in which packets arrive at the destination, \
each one must contain 2 identifiers: one that uniquely identifies each image, and one that \
uniquely identifies the order of the packet in its image. \
Using these 2, the image can be safely reconstructed from the UDP packets. \
While the second identifier (for packets in an image) can be simply an index, the first \
identifier (for images) was chosen to be the unix timestamp in milliseconds at which the \
image was taken.

The second path that the image needs to go through is from the proxy server to \
multiple (web) clients connected to that server. \
Since this communication is initiated by the server, it can no longer use \
UDP packets, and thus TCP protocol will have to be used. \
The best way to send data in a continuous stream from server to multiple web \
clients is using web sockets, which are supported by most browsers.

%\subsection{Person Detection}
%\label{subsec:person-detection}
%Person detection consists of 2 phases: full person detection, and face detection. % maybe test just face detection
%This is due to the fact that it's easier to detect and track a person than to detect and track a face.
%For person detection I have chosen neural networks
