
\chapter{Analysis and Theoretical Foundation}
\label{ch:analysis}

%Together with the next chapter takes about 60\% of the whole paper
%
%The purpose of this chapter is to explain the operating principles of the implemented application.
%Here you write about your solution from a theory standpoint - i.e. you explain it and you demonstrate its theoretical properties/value, e.g.:
%\begin{itemize}
% \item used or proposed algorithms
% \item used protocols
% \item abstract models
% \item logic explanations/arguments concerning the chosen solution
% \item logic and functional structure of the application, etc.
%\end{itemize}
%
%{\color{red} YOU DO NOT write about implementation.
%
%YOU DO NOT copy/paste info on technologies from various sources and others alike, which do not pertain to your project.
%}

%\section{Title}
%\section{Other title}
\section{Algorithms}
\label{sec:analysis-algorithms}
 The project relies on several different algorithms for achieving its mission to detect and track people.
 The most notable algorithms are for image transmission, person detection, object tracking and face recognition.

\subsection{Image transmission}
\label{subsec:image-transmission}
 I needed to devise an algorithm that would allow to one source to send multiple images to several receivers that are
 not on the same network and at considerable distances, and at the same time having a delay as small as possible.

 According to [1], \ %todo: insert bib ref
different low level heuristics behave differently in different environments, with some working better than others. \
In order to rank low level heuristics, 2 evaluation functions are used: \textbf{competence} and \textbf{affinity}. \
Competence describes how well a heuristic works on its own. \
Affinity is a relationship between 2 low level heuristics \
and describes how well a low level heuristic works when being applied after another low level heuristic. \
This way, \
when choosing a new low level heuristic to apply on the solution, we can use either competence or affinity to pick \
one. \
However, we also need to take into account the time passed since the low level heuristic was last used, otherwise \
some low level heuristics would never be used. \
The formula used for choosing a low level heuristic by competence is \
the following:

\( P(h) = \frac{C(h)}{\sum_{h_i \epsilon H} C(h_i)} * \Delta T_h \)

Where $P(h)$ is the probability of choosing heuristic $h$, $C(h)$ is the competence is heuristic h and $H$ is the set \
of all low level heuristics and $\Delta T_h$ is the time spent since the last time heuristic $h$ was last used. \
This way, we ensure that all heuristics are used, and not just 2 or 3 that behave the best.

The affinity formula used for choosing a heuristic is the following:

\( P(h) = \frac{Aff(h_\alpha, h)}{\sum_{h_i \epsilon H} Aff(h_\alpha , h_i)} * \Delta T_h \)

Where $P(h)$ is the probability of choosing heuristic $h$, $h_\alpha$ is the heuristic that was last applied, \
$Aff(h_\alpha , h)$ is the affinity between heuristics $h_\alpha$ and $h$ and $\Delta T_h$ is the time spent since \
heuristic $h$ was last used.

Both the competence and the affinity must be initialized in a pre-run phase and then must be updated after each \
heuristic is applied. \
In order to initialize them, we need several sequences of heuristics that have behaved well or \
simply better than other sequences. \
We shall note the set of such sequences $\Theta$. \
The formula for initializing the competence is the following:

\( C(h_A) = \sum_{H \epsilon \Theta} \sum_{h_i \epsilon H} h_A == h_i ? 1 : 0 \)

Basically, the above formula counts the number of times heuristic $h_A$ appears in the set of good heuristic \
sequences. \
The formula for initializing affinity is the following:

\(  Aff(h_A, h_B) = \sum_{H \epislon \Theta} \sum_{i=1}^{len(H) - 2} \sum_{j=i+1}^{len(H) - 1} h_A == h_i \wedge \
h_B == h_j ? \frac{1}{j - i} : 0 \) %todo: fix &&

Thus, the affinity between 2 heuristics is maximum when they are placed one after another (are adjacent) and decrease \
as the distance between them grows.

In order to choose between competence and affinity when deciding which heuristics to use next, we can define a \
constant $\alpha,  0 < \alpha < 1$, and then generate a random number between 0 and 1. \
If the number is smaller than $\alpha$, we will use competence, otherwise affinity. %todo: validate statement with alpha


\subsection{Simulated Annealing Meta Heuristic}
\label{subsec:analysis-simulated-annealing}
The term \textit{Simulated Annealing} originates from metallurgy, where \textit{annealing} refers to a technique that \
consists of heating a material above its melting point and then gradually cooling it in order to reduce its defects \
and increase the size of its crystals.
In Computer Science, \textit{Simulated Annealing} is a simulation technique used to detect global optimum in an \
environment with many local optimums.

According to \cite{springer1}, \ %todo: fix citation and complete this


\subsection{Proposed Hyper Heuristic}
\label{subsec:analysis-proposed-hh}
The proposed algorithm is a combination of the choice function hyper heuristic and the simulated annealing \
meta-heuristic. \
The algorithm steps can be found below: %todo: add label to algorithm

\noindent\rule{\textwidth}{1pt}
\begin{enumerate}[itemsep=1pt, parsep=1pt, topsep=1pt]
    \item Generate an initial solution of the optimization problem, $sol_{domain}$, randomly
    \item Generate a set of low level heuristics sequences randomly
    \item Apply each sequence of the low level heuristic on the same $sol_{domain}$
    \item \begin{enumerate}[itemsep=1pt, parsep=1pt, topsep=1pt]
              \item The best resulting solution will be $sol_{domain\_opt}$
              \item Save the best n sequences
    \end{enumerate}
    \item Using the best n sequences, initialize the competence and affinity using the equations presented above. %todo: link competence and affinity
    \item Initialize temperature T
    \item While the temperature T is above a certain threshold:
    \item \begin{enumerate}[itemsep=1pt, parsep=1pt, topsep=1pt]
              \item Pick a low level heuristic h using either competence or affinity
              \item Apply h on $sol_{domain}$; the resulting solution will be $sol_{domain1}$
              \item If $sol_{domain1}$ is better than $sol_{domain}$:
              \begin{enumerate}[itemsep=1pt, parsep=1pt, topsep=1pt]
                  \item Update the competence and affinity of heuristic h
                  \item $sol_{domain}$ will take the value of $sol_{domain1}$
                  \item if $sol_{domain1}$ is better than $sol_{domain\_opt}$, than $sol_{domain\_opt}$ will take the value of $sol_{domain1}$
              \end{enumerate}
              \item If $sol_{domain1}$ is not better than $sol_{domain}$
              \begin{enumerate}[itemsep=1pt, parsep=1pt, topsep=1pt]
                  \item $sol_{domain}$ will take the value of $sol_{domain1}$ with the probability $P = e^{\frac{-\Delta E}{T}}$
              \end{enumerate}
              \item Update the temperature $T = T - \Delta T$
    \end{enumerate}
\end{enumerate}
\noindent\rule{\textwidth}{1pt}

As can be seen from the algorithm, the choice function hyper-heuristic is used to select the next heuristic to be \
applied on the solution, while simulated annealing is used to sometimes pick a worse solution in order to escape \
local minimums. \
The main idea of this algorithm is not to obtain the best solution possible, but a good enough \
solution in a reasonable time.


%todo: complete here
\subsection{Abstract Models}
\label{subsec:analysis-models}
The search space is represented by meals. \
These are characterized by multiple properties:
\begin{itemize}
    \item \textbf{Name} Name of the meal
    \item \textbf{Type} Must be one of the following: \textit{Breakfast, First Snack, Lunch, Second Snack, Dinner}
    \item \textbf{Distributor} A distribution company with an address from which meals will be delivered
    \item \textbf{Nutritional components}
    \item \textbf{Food Items} The raw ingredients the meal is made from.
    \item \textbf{Reliability} is a combination of aspect, taste and smell
    \item \textbf{Price}
    \item \textbf{Time} Time it takes to cook and deliver the meal
\end{itemize}
A menu (recommendation) is made of 5 meals of different types: breakfast, first snack, lunch, second snack and dinner.




\subsection{Low Level Heuristics}
\label{subsec:analysis-llh}
The proposed hyper heuristic uses 9 different low level heuristics, most of which were taken from genetic algorithms:
\begin{itemize}
    \item \textbf{Optimum Single Point Mutation} for each type of meal, it retrieves a meal from the database of the \
same type. If the new meal is better than the current meal, than it replaces the current meal and then the heuristic \
stops. If no new meal is better than its current equivalent, the current menu remains unchanged.
    \item \textbf{Optimum Single Point Crossover} uses the optimum domain solution as a reference solution. For each \
type of meal, it checks if the optimum solution meal is better than the current meal. If so, it replaces the current \
meal with the optimum solution meal and then it stops. If no meal from the optimum solution is better than its current \
equivalent, than the current meal remains unchanged.
    \item \textbf{Optimum Multiple Point Mutation} is similar to the optimum single point mutation. the only \
difference is that it doesn't stop once one meal has been replaced. If all new meals retrieved from the database are \
better than their current equivalents, than all current meals will be replaced.
    \item \textbf{Optimum Multiple Point Crossover} is similar to the optimum single point crossover, the only \
difference being the same difference as between single point and multiple point mutation
    \item \textbf{Random Single Point Mutation} replaces one random meal with a new one from the database. The new \
meal doesn't have to be better than the current one.
    \item \textbf{Random Single Point Crossover} replaces one random meal with a meal of the same type from the optimum \
solution.
    \item \textbf{Random Multiple Point Mutation} replaces all current meals with new one from the database
    \item \textbf{Random Multiple Point Crossover} replaces the current menu with the optimum menu
    \item \textbf{Memory Based Mutation} each time a meal is replaced with another meal, the difference in score \
between the before and after menu is saved in a memory. This heuristic uses that memory to try and replace all meals \
from the current solution with meals from memory that have yielded the largest score increase
\end{itemize}

%todo: complete this
\section{User Profile}
\label{sec:analysis-user-profile}
The user profile consists of information about allergies to food items and nutrient intake values.

%todo: retun here
\section{Fitness Function}
\label{sec:analysis-fitness}
This Fitness function is ued to evaluate the quality of a menu or of a meal from a menu. \
It is inverse proportional with te quality of a meal. \
Thus, the better the meal/menu, the lower the fitness. \
When computing the fitness of a menu, several factors must be taken into consideration:
\begin{itemize}
    \item \textbf{Nutritional values} for each meal type (breakfast, lunch, dinner, 2 snacks)
    \item \textbf{Price} the price of the meal should not be much higher than the consumer desired price
    \item \textbf{Time}
    \item \textbf{Reliability}

\end{itemize}

\section{Structure of Application}
\label{sec:analysis-structure}
The application consists of 3 main components:
\begin{itemize}
    \item \textbf{Data Access Component} Is used to retrieve users, users profile and meals from database. This is a \
            component used by multiple algorithms used to generate menus: choice function, cuckoo search.
    \item \textbf{Main Algorithm} This is the hyper-heuristic algorithm used to generate a menu
    \item \textbf{Web component} that all users (elders, nutritionists, distributors) will interact with.
\end{itemize}
