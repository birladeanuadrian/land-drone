\chapter{Detailed Design and Implementation}
\label{ch:implementation}

% todo: detaliere fiecare chema

% results: rezultate intermediare, rezultate finale (hardware timp de raspuns, timp duratie transmitere, timp procesare imagine, tabele acuracy, procentaj detectie fete, procentaj recunoastere fete)
% pe cd: cod + documentatie
% gdpr compliance

\section{Introduction}
\label{sec:implementation-introduction}
This project contains 4 different components that are deployed in different \
places:
\begin{enumerate}
    \item the robot application (deployed and running on the actual robot)
    \item the proxy server that acts as an intermediary between the user \
            controlling the robot and the robot; \
            it runs in a kubernetes \
            cluster in the cloud (GKE, more precisely)
    \item an angular web application that is used to control the robot; \
            it is \
            deployed in the same cloud as the proxy server, but runs in the \
            user's browser
    \item the algorithm used to split an image into several UDP-ready packets \
            and to reconstruct the image from said packets; \
            the algorithm is published as a public package that is imported by \
            both the robot application and the web application
\end{enumerate}

Each component's detailed design and implementation will be detailed below.

\section{Robot Application}
\label{sec:robot-application}
I have chosen to run the robot application on a Raspberry PI 3 both \
because of the support for high-level development languages (Python, \
NodeJS, as opposed to VHDL/Verilog), and because of the support for \
third party modules/application (in this instance, RabbitMQ).

Physically, the robot consists of a platform with 4 wheels, a Raspberry PI, \
batteries and a camera, all connected with wires.

From a software point of view, it consists of 2 independent modules communicating \
with one another via RabbitMQ queues. \
The modules are engine control and external comms. \
The engine control module controls the 4 wheels independently and can make the \
robot go forward, reverse and steer. \
The external comms has 2 roles: capture video from camera to transmit it to the \
server, and listen for commands from the remote server in order to transmit them \
to the engine control via RabbitMQ.

Having separate processes leads to low couping and ensures that changes in one \
component do not affect other components.

%Together with the previous chapter takes about 60\% of the paper.
%
%The purpose of this chapter is to document the developed application such a way that it can be maintained and \
%developed later. \
%A reader should be able (from what you have written here) to identify the main functions of the application.
%
%The chapter should contain (but not limited to):
%\begin{itemize}
%    \item a general application sketch/scheme,
%    \item a description of every component implemented, at module level,
%    \item class diagrams, important classes and methods from key classes.
%\end{itemize}
