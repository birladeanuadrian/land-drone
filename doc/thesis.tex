%%%%%%%%%%%%%%%%%%%%%%%%%%%%%%%%%%%%%%%%%%%%%%%%%%%%%%%%%%%%%%%%%%%%%%%%%%%%%
%%%
%%% File: thesis.tex, version 1.9, May 2016
%%%
%%% =============================================
%%% This file contains a template that can be used with the package
%%% cs.sty and LaTeX2e to produce a thesis that meets the requirements
%%% of the Computer Science Department from the Technical University of Cluj-Napoca
%%%%%%%%%%%%%%%%%%%%%%%%%%%%%%%%%%%%%%%%%%%%%%%%%%%%%%%%%%%%%%%%%%%%%%%%%%%%%

\documentclass[12pt,a4paper,twoside]{report}         
\usepackage{cs}              
\usepackage{times}
\usepackage{graphicx}
\usepackage{latexsym}
\usepackage{amsmath,amsbsy}
\usepackage{amssymb}
\usepackage[matrix,arrow]{xy}
\usepackage[T1]{fontenc}
\usepackage{ae,aecompl}
%\usepackage{shortcut} %definitii pentru diacritice; 
\usepackage{amstext}
\usepackage{graphics}
\usepackage{ae,aecompl}
\usepackage{algorithm}
%\usepackage{algorithmic}
\usepackage{color}
\usepackage{enumitem}

% \mastersthesis
\diplomathesis
% \leftchapter
\centerchapter
% \rightchapter
\singlespace
% \oneandhalfspace
% \doublespace

\newcommand{\applicationTitle}{Land Drone}

\renewcommand{\thesisauthor}{Firstname LASTNAME}    %% Your name.
\renewcommand{\thesismonth}{June}     %% Your month of graduation.
\renewcommand{\thesisyear}{2018}      %% Your year of graduation.
\renewcommand{\thesistitle}{LICENSE THESIS TITLE} 
\renewcommand{\thesissupervisor}{scientific title Firstname LASTNAME}
\newcommand{\department}{\bf FACULTY OF AUTOMATION AND COMPUTER SCIENCE\\
COMPUTER SCIENCE DEPARTMENT}
\newcommand{\thesis}{LUCRARE DE LICEN'T'A}
\newcommand{\utcnlogo}{\includegraphics[width=15cm]{img/tucn.jpg}}

\newcommand{\uline}[1]{\rule[0pt]{#1}{0.4pt}}
%\renewcommand{\thesisdedication}{P\u{a}rin\c{t}ilor mei}

\begin{document}
%\frontmatter
%\pagestyle{headings}

\newenvironment{definition}[1][Defini\c{t}ie.]{\begin{trivlist}
\item[\hskip \labelsep {\bfseries #1}]}{\end{trivlist}}



%\thesistitle                    %% Generate the title page.
%\authordeclarationpage                %% Generate the declaration page.

\pagenumbering{arabic}
\setcounter{page}{4}



\begin{center}
\utcnlogo

\department

\vspace{4cm}

{\bf \thesistitle} %LICENSE THESIS TITLE}

\vspace{1.5cm}

LICENSE THESIS

\vspace{6cm}

Graduate: {\bf Adrian BIRLADEANU}

Supervisor: {\bf \thesissupervisor}

\vspace{3cm}
{\bf \thesisyear}
\end{center}

\thispagestyle{empty}
\newpage

\begin{center}
\utcnlogo

\department

\end{center}
\vspace{0.5cm}

%\begin{small}
\begin{tabular}{p{7cm}p{8cm}}
 %\hspace{-1cm}& APPROVED,\\
 \hspace{-1cm}DEAN, & HEAD OF DEPARTMENT,\\
 \hspace{-1cm}{\bf Prof. dr. eng. Liviu MICLEA} & {\bf Prof. dr. eng. Rodica POTOLEA}\\  
\end{tabular}
 
\vspace{2cm}

\begin{center}
Graduate: {\bf \thesisauthor}

\vspace{1cm}

{\bf \thesistitle}
\end{center}

\vspace{1cm}

\begin{enumerate}
 \item {\bf Project proposal:} {\it Short description of the license thesis and initial data}
\item {\bf Project contents:} {\it (enumerate the main component parts) Presentation page, advisor's evaluation, title of chapter 1, title of chapter 2, ..., title of chapter n, bibliography, appendices.}
\item {\bf Place of documentation:} {\it Example}: Technical University of Cluj-Napoca, Computer Science Department
\item {\bf Consultants:}
\item {\bf Date of issue of the proposal:} November 1, 2016
\item {\bf Date of  delivery:} February 21, 2018 {\it (the date when the document is submitted)}
  \end{enumerate}
\vspace{1.2cm}

\hspace{6cm} Graduate: \uline{6cm} 

\vspace{0.5cm}
\hspace{6cm} Supervisor: \uline{6cm} 
%\end{small}

\thispagestyle{empty}


\newpage
$ $
%\begin{center}
%\utcnlogo

%\department
%\end{center}

\thispagestyle{empty}
\newpage

\begin{center}
\utcnlogo

\department
\end{center}

\vspace{0.5cm}

\begin{center}
{\bf
Declara\c{t}ie pe proprie r\u{a}spundere privind\\ 
autenticitatea lucr\u{a}rii de licen\c{t}\u{a}}
\end{center}
\vspace{1cm}



Subsemnatul(a) \\
\uline{14.8cm}, 
legitimat(\u{a}) cu \uline{4cm} seria \uline{3cm} nr. \uline{4cm}\\
CNP \uline{9cm}, autorul lucr\u{a}rii \uline{2.8cm}\\
\uline{16cm}\\
\uline{16cm}\\
elaborat\u{a} \^{\i}n vederea sus\c{t}inerii examenului de finalizare a studiilor de licen\c{t}\u{a} la Facultatea de Automatic\u{a} \c{s}i Calculatoare, Specializarea \uline{7cm} din cadrul Universit\u{a}\c{t}ii Tehnice din Cluj-Napoca, sesiunea \uline{4cm} a anului universitar \uline{3cm}, declar pe proprie r\u{a}spundere, c\u{a} aceast\u{a} lucrare este rezultatul propriei activit\u{a}\c{t}i intelectuale, pe baza cercet\u{a}rilor mele \c{s}i pe baza informa\c{t}iilor ob\c{t}inute din surse care au fost citate, \^{\i}n textul lucr\u{a}rii \c{s}i \^{\i}n bibliografie.

Declar, c\u{a} aceast\u{a} lucrare nu con\c{t}ine por\c{t}iuni plagiate, iar sursele bibliografice au fost folosite cu 
respectarea legisla\c{t}iei rom\^{a}ne \c{s}i a conven\c{t}iilor interna\c{t}ionale privind drepturile de autor.

Declar, de asemenea, c\u{a} aceast\u{a} lucrare nu a mai fost prezentat\u{a} \^{\i}n fa\c{t}a unei alte comisii de examen de licen\c{t}\u{a}.

\^{I}n cazul constat\u{a}rii ulterioare a unor declara\c{t}ii false, voi suporta sanc\c{t}iunile administrative, respectiv, \emph{anularea examenului de licen\c{t}\u{a}}.

\vspace{1.5cm}

Data \hspace{8cm} Nume, Prenume

\vspace{0.5cm}

\uline{3cm} \hspace{5cm} \uline{5cm}

\vspace{0.5cm}
\hspace{9.4cm}Semn\u{a}tura

\thispagestyle{empty}

\newpage


%\listoftables
%\listoffigures

%\clearpage 
%\newpage

%\begin{comment}
\include{guideline} 
%\end{comment}

\newpage

\tableofcontents
\newpage




%\subsection{Subsection}
%%Each table used in this document is labeled as Table x.y, where x represents the chapter number, and y shows the table number within the current chapter. Leave a blank line between and after each table, relative to the adjacent paragraphs (table~\ref{table:nonlin}).
%
%\begin{table}[ht]
%\caption{Nonlinear Model Results}
%\centering                          % tabel centrat
%\begin{tabular}{|c|c|c|c|}          % 4 coloane centrate
%\hline\hline                        % linie orizontala dubla
%Case & Method\#1 & Method\#2 & Method\#3 \\ [0.5ex]   % inserare tabel
%%heading
%\hline                              % linie orizontal simpla
%1 & 50 & 837 & 970 \\               % corpul tabelului
%2 & 47 & 877 & 230 \\
%3 & 31 & 25 & 415 \\[1ex]           % [1ex] adds vertical space
%\hline
%\end{tabular}
%  % titlul tabelului
%\label{table:nonlin}                % \label{table:nonlin} introduce eticheta folosita pentru referirea tabelului in text; referirea in text se va face cu \ref{table:nonlin}
%\end{table}
%
%Each figure used in the document must be cited within the text (ex: in figure x.y the system components are presented... ) and labeled. The labeling must be as Figure x.y where x represents the chapter number, and y shows the number of the figure within the current chapter.
% E.g.: figure \ref{fig:imag}.
%
%\begin{figure}[ht]
%    \centering
%\includegraphics[]{img/test.jpg}
%    \caption{The figure`s name}
%    \label{fig:imag}
%\end{figure}
%
%Each chapter must start on a new page.

\usepackage{graphicx}


\chapter{Introduction - Project Context}
\label{ch:introduction}
\pagestyle{headings}

\section{Project context}
\label{sec:introduction-context}

\subsection{General Context}
\label{subsec:introduciton-general-context}
With the increase
In the last century, because of advances in technology and medicine, the average lifespan has increased to a global \
average of 70.5 years, with and European average of about 80.2 years\
\footnote{According to https://en.wikipedia.org/wiki/List\_of\_countries\_by\_life\_expectancy}, with more details in \
\ref{table:lifeExpectancy}. \
This in turn has led to people taking better care for themselves at older ages. \
There are several ways to improve life quality and extend life expectancy. \
Most doctors agree that one of the most important ones is a correct diet.

\begin{table}[ht]
    \caption{Average Life Expectancy in Europe}
    \centering                          % tabel centrat
    \begin{tabular}{|c|c|c|}          % 3 coloane centrate
        \hline\hline                        % linie orizontala dubla
        Area & Male & Female \\ [0.5ex]   % inserare tabel
        %heading
        \hline                              % linie orizontal simpla
        Europe & 75 & 81 \\               % corpul tabelului
        Western Europe & 79 & 84  \\
        Southern Europe & 79 & 84 \\           % [1ex] adds vertical space
        Northern Europe & 79 & 83  \\
        Eastern Europe & 68 & 78 \\[1ex]
        \hline
    \end{tabular}
    \label{table:lifeExpectancy}
\end{table}

\subsection{Specific Context}
\label{subsec:introduction-specific-context}

As people get older, their biological functions begin to fail, and they start to experience health issues. \
According to \
the World Health Organization\footnote{http://www.who.int/news-room/fact-sheets/detail/mental-health-of-older-adults},\
an estimated 15\% of all adults over 60 suffer from some sort of mental disorder. \
Some of the most common are depression, anxiety and dementia. \
This can lead to either a lack or an increased appetite, either of which can lead to a chemical \
imbalance in the body, causing even more health problems. \
Another common issue among elder people is the degradation of \
senses. \
People tend to lose the sense of smell, taste, sight, which means they cannot be fully aware of what they eat.\
Without strict control, this can lead to people eating foods that may be harmful to them without even knowing (such as \
people having heart problems with too salty foods and not sensing it).


\subsection{Nutrients}
\label{subsec:introduction-nutrients}
There are a few nutrients that are very important to the organism, and we will present them briefly.

Proteins are one of these nutrients. \
For instance, hair and nails are mostly made of protein, along with muscular tissue. \
They are required to build enzymes, hormones and other chemicals found in the human body. \
For active people, theoretically there is no upper intake limit. \
However, out target group is make of elder people, and an important fact \
that needs to be taken into account is that proteins contain about 4 calories/100g. \
On the other hand, a lack of proteins \
can cause diarrhea, fatigue, loss of muscle mass, lethargy, reduced immunity and can ultimately lead to death.

Lipids are another important group of nutrients. \
They naturally occur in fats, monoglycerids, fat-soluble vitamins and others. \
They are mainly used to store energy, as part of the membrane that surrounds cells, as building blocks for \
several hormones and as components of the nervous system. \
Excess consumption of lipids can cause multiple cardio-vascular problems and obesity. \
Underconsumption can also lead to health issues.

Carbo-hydrates are also an important nutrient, as they are a source of energy. \
However, they are not essential to humans,as people can take most of their energy from proteins and fats. \
Excess consumption often leads to type 1 diabetes.

Another major group of important nutrients are minerals. \
Calcium is generally important for health, as it is being used by the nervous system, muscles, heart and bones. \
Over-consumption over a long period of time can lead to a higher risk of kidney stones, while lack if calcium in \
the body can lead to osteoporosis (thinning of bone tissue and decreased bone density).

Iron is another essential mineral. \
It is one of the components of hemoglobin, which is used by red blood cells to carry oxygen throughout the body. \
About two thirds of the iron in the human body is found in hemoglobins. \
Lack of iron can lead to fatigue, dizziness and lowered immunity. \
Excess iron can build up in organism, leading to diseases like liver cancer, cardiac arrhythmia, diabetes, \
Alzheimer's, bacterial and viral infections.

Sodium is another important mineral in health. \
It plays a part in muscle contraction and nerve impulse conduction. \
Lack sodium causes cellular functions and neural communication to stop. \
Excess iron builds up in blood vessels, causing over time high blood pressure, heart attack and stroke.

Another important group of nutrients is made of vitamins. \
Vitamin A is naturally occurs in many foods. \
It plays a part vision, immune system and reproduction. \
Lack of it can cause sight issues, while excess can lead to dizziness, headaches, coma or even death.

The B-vitamin family plays a part in turning food into energy and metabolizing fats. \
This family contains 8 vitamins (B1-B8). \
B6 is used in the digestive, immune, muscular, cardiovascular and nervous system.Lack of it can cause anemia, \
rashes or swollen tongue, depression, weak immune system and confusion.
Excess can cause nerve damage.

Vitamin C is required for normal growth and development. \
However, consuming more than 2000mg/day can lead to stomach upset and diarrhea. \
Lack of it can cause symptoms of deficiency, such as bleeding gums, anemia, weaker immune system, \
weight gain, painful or swollen joints.

Vitamin D is also important for good overall health. \
It plays a part in the function of muscles, lungs, heart, brain and immune system. \
It is required for the absorption of calcium. \
Lack of vitamin D over longer periods of time can lead to \
fragile bones in adults, colon cancer, prostate cancer, depression, heart diseases, weight gain and more. \
Excess can cause kidney stones, heart problems, confusion, nausea.


\subsection{Motivation}
\label{subsec:introduction-motivation}
Europe is dealing with an increasingly older population. \
Lifespan is continuously increasing and several European populations are dealing with major problems concerning \
demographic changes and the transition to a much older population structure.


\begin{figure}[ht]
    %\centering
    \label{fig:eurostat1}
    \includegraphics[width=15cm, height=60cm,keepaspectratio]{img/eurostat1.png}
    \caption{Population pyramid in 2080 compared to 2017, according to Eurostat}
\end{figure}

According to ~\cite{eurostat1}, as can be seen in figure \ref{fig:eurostat1}, current predictions estimate \
that the percentage of elder people will increase. \
Estimates put the number of people aged 65 years or more as approximately 29.1\% of \
the total population by 2080, compared to the current value of approximately 19.4\%.

Although it may seem that a longer life is a good thing, and indeed the advances in medicine and technology from \
the last century have granted us longer lives, an important issue remains the distinction between lifespan and the \
quality of life. \
According to a study made by the University of South Carolina between 1970 and 2010,
\footnote{https://news.usc.edu/98750/americans\-living\-longer\-with\-disability\-or\-health\-issues\-study\-shows/},\
although the total lifespan increased in this period, so did the time spent with a disability. \
This means that people need to pay more attention to living healthy. \
And in order to do that, people need to eat healthy first.


%\usepackage{graphicx}


\chapter{Project Objectives and Specifications}
\label{ch:specification}

\section{Problem specification}
\label{sec:specification-specification}

Internet-controlled drones can take several forms and serve multiple purposes. \
This specific drone is a land drone (car) that streams video-only, not audio for now, and accepts \
commands from a web interface. \
Because 5G networks aren't accessible except via high-end smartphones, the robot will be connected \
to the internet via a 4G LTE mobile modem.

I have chosen a land drone because it is both easier and more affordable to build and test. \
However, most components (cloud server, web interface, drone - server communication) can be reused \
in order to build other types of drones (naval and flying drones).

Additionally, the drone will have person detection capabilities in order to \
meet certain security-related use cases, such as monitoring remote locations.

The current state-of-the-art in remote controlled drones is met on military drones. \
According to \
\footnote{https://www.forbes.com/sites/sebastienroblin/2019/09/30/dont\-just\-call\-them\-drones\-a\-laypersons\-guide\-to\-military\-unmanned\-systems\-on\-air\-land\-and\-sea/\#76c957e62b00}
the most popular control mechanisms are by radio and by satellite uplink. \
While the radio control requires a certain proximity of the drone to the operator (i.e., 100 miles), \
drones controlled via satellite uplink introduce a certain latency (image transmitted from a drone \
in Afghanistan may take up to 1.2 seconds to reach the operator in the US). \
High-Speed internet might provide a third control mechanism, that could remove the range limitation \
of the radio and decrease the latency of current satellite uplinks.

Possible commercial use cases for this drone are wildlife observation and perimeter patrol.


\section{General Objectives}
\label{sec:specification-objectives}
The primary objective of this project is to create a drone that can be remotely controlled from \
half-way around the world.

\subsection{Use Cases}
\label{subsec:use-cases}
The project will need to meet the following use cases:

\subsubsection{\textbf{Login}}

\textbf{Primary Actor:} Drone Operator

\textbf{Stakeholders and Interests:}
\begin{enumerate}
    \item \textit{Drone Operator}: wants to be the sole person who controls \
            the drone at a given moment
    \item \textit{Drone owner}: wants to limit access to the drone video feed \
            and control to authorized personnel only
\end{enumerate}

\textbf{Postcondition:} The operator is granted access to the land drone

\textbf{Main Success Scenario}
\begin{enumerate}
    \item the operator goes to the drone url
    \item the operator inserts the username and password
    \item the cloud server validates the username and password
    \item the cloud server validates that no one else controls the drone
    \item the cloud server gives the operator access to the drone video \
            feed and controls
\end{enumerate}

\textbf{Extensions}
\begin{itemize}
    \item 3. The username or password is incorrect
        \begin{itemize}
            \item an error message is shown
            \item video feed and controls are disabled until the operator \
                    reloads the page
        \end{itemize}
    \item 4. Another operator already controls the drone
        \begin{itemize}
            \item the current operator is given access only to the video \
                    feed, while the first operator retains access to the controls
        \end{itemize}
\end{itemize}

\subsubsection{\textbf{View Drone Footage}}
\textbf{Primary Actor:} Drone Operator

\textbf{Stakeholders and Interests:}
\begin{enumerate}
    \item \textit{Drone Operator:} wants to view a live video feed from the drone
    \item \textit{Drone Operator and Owner} want to view the drone footage at a later date
\end{enumerate}

\textbf{Preconditions:} Drone operator has logged in and is the sole operator for the drone

\textbf{Main Success Scenario}
\begin{enumerate}
    \item Drone operator goes to the drone page
    \item Drone operator clicks on the \textbf{Start video feed} button
    \item The drone starts transmitting video feed
    \item The frontend shows the live video feed
\end{enumerate}

\textbf{Extensions}
\begin{itemize}
    \item The drone operator wants to save the video feed so that he can watch it at a later date
        \begin{itemize}
            \item After clicking on the \textbf{Start video feed} button, the operator clicks on the \
                    \textbf{Record feed} button
            \item The system starts recording the video feed
            \item The operator clicks on the \textbf{Stop recording} button
            \item The operator click on the \textbf{Download recording button}
            \item The system downloads the recording
        \end{itemize}
    \item The drone operator wants people to be highlighted
            \begin{itemize}
                \item After clicking on the \textbf{Start video feed}, but before clicking on the \
                        \textbf{Record feed} button, the operator presses on the \textbf{Highlight people} button
                \item The system shows the video feed, highlighting people and writing the names of the
                        known ones
            \end{itemize}
    \item A second operator wants to view the real time footage
            \begin{itemize}
                \item The second operator logs in
                \item The second operator views the same footage as the first
                \item The system disables all actions for the second operator (start recording, highlight people)
            \end{itemize}
\end{itemize}

\subsubsection{Control the drone}

\textbf{Primary Actor:} Drone Operator

\textbf{Stakeholders and Interests:}
\begin{enumerate}
    \item \textit{Drone operator} wants to move the drone in specific directions in order to meet his objectives
\end{enumerate}

\textbf{Preconditions:} Drone operator has logged in

\textbf{Main Success Scenario}
\begin{enumerate}
    \item The drone operator starts video feed and waits for it
    \item The drone operator controls the drone using the keyboard arrow keys
\end{enumerate}

\textbf{Extensions}
\begin{itemize}
    \item A second operator enters the drone page while the first operator control the drone
            \begin{itemize}
                \item The system disables the controls for the second operator
            \end{itemize}
\end{itemize}

The presented use cases can be applied in the following fields:
\begin{enumerate}
    \item \textbf{Wildlife observation} the robot can be used to observe wild animals from a distance in areas with \
        decent internet coverage (minimum 7Mbps upload speed)
    \item \textbf{Perimeter patrol} the robot can be used to patrol remote open areas where it's not possible to \
        mount security cameras (forests, borders)
\end{enumerate}


\section{Functional Requirements}
\label{sec:functional-requirements}

In software engineering, a functional requirement is a function of a system or its components. \
Such a function consists of a specific set of inputs, a behavior and an output. \
Functional requirements describe what the system \
is expected to do in order to accomplish its purpose and reach the desired results.

The functional requirements I have set out to achieve are mostly based on the use cases presented above. \
They are the following:
\begin{enumerate}
    \item live video feed from the drone to a web interface
    \item full control of the drone from a web interface
    \item recording capabilities for the live video feed
    \item option to highlight people
    \item limit of one active operator, with multiple passive operators having access to just the video feed, not the commands
\end{enumerate}

\section{Non-Functional Requirements}
\label{sec:non-functional-requirements}

In software engineering, a non-functional requirement refers to a system's quality characteristics and attributes, \
unlike functional requirements which refer to a system's functions. \
Non-functional requirements are used to \
evaluate all operations of the system, not just a specific component or behavior. \
They can be split into 2 \
categories: execution requirements (like security, performance) and evolution requirements (like scalability,
testability, extensibility). \
In the following subsections we will present in detail all the non-functional requirements.

\subsection{Performance}
\label{subsec:specification-performance}
Performance is one of the most important requirements of the system and is measured in several possible ways:
\begin{itemize}
    \item Video latency
    \item Video FPS
    \item Response time to commands
\end{itemize}

The video latency is measured in milliseconds since the image displayed was taken. \
Since the project won't be using 5G internet networks, but 4G LTE, I set out to achieve an average of \
15FPS with a latency of 500ms and a command response time of 100ms.

\subsection{Deployment}
\label{subsec:deployment}
The deployment will consist of hardware deployment (the drone itself plus its configuration) and software \
deployment (the control UI and other proxy and processing servers). \
The first step is the software deployment, which will be done in a kubernetes cluster. \
This can be done with relative ease. \
THe second step is the drone deployment, which includes a step for configuring the drone (writing the \
ip address to which video feed will be sent and the ip address from which commands will be received).

\subsection{Scalability and extendability}
\label{subsec:specification-scalability}
Scalability of the system refers to the ease with each the system can accommodate an increased \
number of users and data from distributors without changing the core of the application. \
On the one hand, the  application is centered on a single drone, with several possible operators. \
On the other hand, the application will be deployed in a Kubernetes cluster, meaning a new instance can easily \
be deployed for more drones.

Extendability refers to the ease with which new functional requirements can be accommodate as \
the number of users and their expectations increase.
One obvious direction of extending is adding more transmission mediums.
The current drone is designed to communicate using 4G LTE networks.
However, users may want to control the drone via Bluetooth or radio waves.
Therefore, the drone should be designed to easily accommodate these requirements.

\subsection{Usability}
\label{subsec:specification-usability}
Usability is the capacity of a system to allow its users to perform their \
tasks safely, effectively and efficiently, all the while offering an \
enjoyable experience.

In this project I aim to make the robot easy to control from any larger \
terminal (tablet, laptop or PC) without imposing certain hardware requirements \
from user devices (like having a state-of-the-art GPU).

\subsection{Security}
\label{subsec:specification-security}
Security is one of the most important non-functional \
requirements and is aimed at protecting the users of the \
system as well as the system itself from non-authorized users \
and malicious intents.
In order to be secure, the application must meet the \
following requirements:
\begin{enumerate}
    \item image transmission must be secured so that a \
        man\-in\-the\-middle cannot see the live footage
    \item drone control should be protected by authentication
    \item the drone should be controlled by only one person \
        at a time in order to prevent conflicting commands \
        which could lead to the drone crashing
    \item attackers must be prevented from overriding the \
        robot's footage with their own footage
\end{enumerate}


\chapter{Bibliographic research}
\label{ch:research}

% https://www.ericsson.com/en/reports-and-papers/white-papers/drones-and-networks-ensuring-safe-and-secure-operations
% https://web.stanford.edu/class/cs231a/prev_projects_2016/deep-drone-object__2_.pdf

\section{Remote Drone Control}
\label{sec:research-remote-drone-control}
The article ~\cite{ericsson1} also approaches the topic of drone control.
According ot it, most current drone use cases cover the situation in which the drone \
operator is in line of sight of the drone and has full control over it, with autonomous \
drones operating outside line of sight gaining more and more importance today.
However, in the future the most common drone use cases will be the ones in which the drone \
operates autonomously outside line of sight without supervision.

\begin{figure}[ht]
    \label{fig:drone-uses}
    %\centering
    \includegraphics[width=15cm, height=50cm,keepaspectratio]{img/drone_uses.png}
    \caption{Drone Uses according to ~\cite{ericsson1}}
\end{figure}

In ~\cite{ericsson1} the author mentions three different options for drone communication and control:
\begin{enumerate}
    \item \textbf{satellite technology} is currently in use today for some drones; \
            its drawbacks include high latency, costs and low throughput
    \item \textbf{dedicated drone terrestrial network} its drawbacks also include high \
            costs and the time it would take to setup an adequate coverage for drones
    \item \textbf{existing terrestrial mobile networks} they have low latency and costs and \
            high throughput; \
            additionally, they have also proven to be secure and robust
\end{enumerate}

Additionally, other requirements can be accomplished using 4G LTE and 5G features.
For instance, drone tracking can be implemented using the mobile positioning service and \
could be queried from the mobile network.

Drone control is also mentioned in ~\cite{forbes1} .
The article also mentions 2 possible control methods that are actively used and that \
partially overlap over those mentioned above. \
These are:
\begin{enumerate}
    \item \textbf{radio waves} these have limited range (an example of 100 miles is given);
    \item \textbf{satellite uplink} it takes footage an average 1.2-second delay to go from \
            a drone in Afghanistan to an operator in Virginia
\end{enumerate}

~\cite{forbes1} also does a classification of drones according to how they are commanded.
Three classes of drones emerge:
\begin{enumerate}
    \item \textbf{remotely piloted} an operator has full control, while the drone can do \
            minimal actions by its own, like avoid crashing
    \item \textbf{semi-autonomous} the drone can perform all or some missions without any \
            human interaction, however an operator exists that can take over control at \
            any time
    \item \textbf{fully autonomous} the drone is engineered to accomplish its mission \
            without any human interaction or supervision; \
            the command link can be missing
\end{enumerate}

\section{Object Detection}
\label{sec:research-object-detection}
In recent years, convolutional neural networks have been used more and more in \
the field of object detection.
According to ~\cite{deepLearning}, A convolutional neural network (CNN) is a \
subclass of multilayer neural networks in which at least one layer applies \
a convolution on its input data instead of matrix multiplication.\
CNNs have seen a rise in use in object detection since 2012, when \
a CNN developed by ~\cite{imagenet} won the ImageNet Large Scale Visual \
Recognition Challenge for the first time.
The CNN managed to reduce the top-1 (actual label differs from predicted label) \
 and top-5 (actual label is not in the predicted top 5 most likely labels) \
error rates from 47.1\% respectively 28.2\% to 37.5\% respectively 17\%.
Additionally, the authors mentioned that the performance was limited by the \
existing hardware and dataset
Ever since, the contest has been won by convolutional neural networks.
By 2016, the top-5 error rate dropped to 3.6\%.
Next I will present some popular object detection neural networks.

% todo: mention SSD vs YOLO
\subsection{YOLOv1}
\label{subsec:research-yolov1-detectors}
The YOLO (You Only Look Once) ~\cite{yolov1} neural network appeared as an alternative \
to other neural networks that required a classifier to be run at differently \
spaced locations across the image.
The YOLO network is a single convolutional neural network that predicts  multiple \
bounding boxes and class probabilities for each box simultaneously.
This approach entails multiple benefits, including increased speed (the network \
doesn't have a complex pipeline) and less background errors ( network takes into \
account the entire image when predicting, not just a single windows, \
leading to less background errors).
However, the accuracy is lesser than for other state-of-the-art networks.
The YOLO network has issues localizing exactly small objects.

The network works by first resizing the image to 448 x 448 dimension, then by dividing \
the image into grid cells of size \textit{S} x \
\textit{S}, with \textit{S} being a network parameter.
If such a grid cell contains the center of an object, then that \
grid cell is in charge of detecting that object.
Each cell predicts \textit{B} boxes and confidence scores for each box.
Additionally, each box also has \textit{C} conditional class probabilities, \
each stating the probability that the detected object is a member of class \
\textit{i}.
The authors created a variant of the original network that had less layers \
(and thus less accuracy), but a higher speed, named \textit{Fast YOLO}.


\subsection{Single Shot MultiBox Detector}
\label{subsec:research-ssd}
Another image detection network is the Single Shot MultiBox Detector \
~\cite{ssd1}.
The network first extracts a feature layer.
By applying a 3 x 3 convolution on that feature layer multiple \textit{k} \
bounding boxes are obtained.
Finally, for each bounding box \textit{c} class scores are computed (with \
\textit{c} being the number of objects the network was trained for) and \
4 relative offsets to the original bounding box shape.

Tests run on the Pascal VOC2007 challenge showed that the SSD network was \
both faster and more accurate than the YOLO network.
The results comparisons, considering resolutions, precision (expressed in mAP - mean \
average precision), and FPS can be seen in table ~\ref{tab:ssd-results}.

\begin{table}[ht]
    \caption{Pascal VOC2007 results}
    \centering
    \begin{tabular}{|c|c|c|c|c|}
        \hline\hline
        Method & mAP \footnote{mean Average Precision} & FPS & \# Boxes & Input resolution  \\
        \hline
        Faster R-CNN & 7.32 & 7 & \~ 6000 & ~1000 x 600 \\
        Fast YOLO & 52.7 & 155 & 98 & 448 x 448 \\
        YOLO & 66.4 & 21 & 98 & 448 x 448 \\
        SSD300 & 74.3 & 46 & 8732 & 300 x 300 \\
        SSD512 & 76.8 & 19 & 24564 & 512 x 512 \\
        \hline\hline
    \end{tabular}
    \label{tab:ssd-results}
\end{table}

%\subsection{YOLOv2}
%\label{subsec:yolov2}
%YOLO v2 (also known as YOLO9000 - since it can detect over 9000 objects classes) \
%is a modification of the initial YOLO algorithm.
%According to the authors, at 67 FPS
%



\section{Object Tracking}
\label{sec:research-tracking}
Some projects use object detection algorithms to detect an initial object, and \
then start applying tracking algorithsm and stop applying detection algorithsm \
because tracking algorithms are faster than the detection algorithms.

In ~\cite{OpencvTracking}, the author present a status of the most \
popular tracking algorithms from opencv.
After analysing multiple algorithms (CSRT, KCF, Boosting, MIL, TLD, \
MedianFlow, MOSSE, GOTURN), the author comes to the following \
conclusions:
\begin{itemize}
    \item \textbf{CSRT (Channel and Spatial Reliability Tracker)}  should be used when higher tracking accuracy \
            is required and the program can tolerate a smaller throughput
    \item \textbf{KCF (kernel correlation filter)} should be used when the program requires greater \
            throughput and faster FPS at the expense of a slightly \
            lower accuracy
    \item \textbf{MOSSE (Minimum Output Sum of Squared Error)} should be used when the speed is of absolute \
            importance
\end{itemize}

In ~\cite{OpencvTracking2} the author also compares the different \
tracking algorithms, adding the FPS for each algorithm.
The KCF tracker reached 409 FPS, but it failed to recover from full \
object occlusion.
The MOSSE tracker reached an astounding 2671 FPS, but it lacked \
behind deep learning based trackers in performance.
Even more, the author mentions that the algorithm looses precision \
even for objects in normal movement.
As for CSRT, it gives higher accuracy, but it reached only 32 FPS.


\section{Related Work}
\label{sec:related-word}

In the paper~\cite{deepDrone}, the authors also attempt to create a drone capable \
of detecting people, with a few differences:
\begin{itemize}
    \item the drone is an airborne one
    \item image processing is done on the drone itself
\end{itemize}

The authors designed an algorithm to detect and track a single person at a time.
They used a Faster RCNN (region based convolutional neural network) to detect a person.
Once the drone detected an object, it stopped applying the neural network on \
new images.
Instead, it applied a KCF (kernel correlation filter) tracking algorithm as longs \
as the object was still in the image.
Once the KCF algorithm detected that the object was no longer in the image, \
the drone stopped applying the tracking algorithm and switched back to the \
detection neural network.
The authors also experimented with a YOLO detector, but they found that although it is \
faster than Faster RCNN, it is less accurate, especially when it comes to small and \
remote people.

The authors tested the tracking and detection algorithms on multiple GPUs, with the \
results shown in th table below:

\begin{table}[ht]
    \caption{Detection and Tracking Results}
    \centering
    \begin{tabular}{|c|c|c|c|}
        \hline\hline
        Hardware Platform & GTX 980 & TX1 & TK1 \\
        \hline
        Power & 150W & 10W & 7W \\
        Detection & 0.17s & 0.6s & 1.6s \\
        Tracking & 5.5ms & 14ms & 14ms \\
        \hline
    \end{tabular}
    \label{table:deep-drone-results}
\end{table}

Since the image processing is done on the drone itself, the power consumption of \
the CPU and GPU must also be taken into consideration in order to determine the \
autonomy of the drone.

In ~\cite{drone3}, the authors consider different possible implementations \
for an object detection algorithm that could be used by UAVs.
They consider an architecture, comprised of 3 parts:
\begin{enumerate}
    \item \textbf{UAV} and associated sensors
    \item \textbf{Cloud computing} VMs/Pods with GPU servers that could do \
            intensive operations
    \item \textbf{Fog Node} - a laptop/smartphone that acts as an \
            intermediary between drone and cloud computing
\end{enumerate}

As for object detection algorithms, the authors consider two different \
types of networks: two-stage and one-stage.
In the two-stage networks, in the first stage a few regions of interest \
are selected, with most of the background being filtered out.
In the second stage, classification is performed for every region \
of interest.
The authors consider R-CNN, Fast-RCNN and Faster R-CNN as possible \
solutions for two-stage networks.
As for one-stage networks, they consider SSD and YOLO.
Although one-stage networks are faster, the authors mention that they \
can cause a larger number of false negatives, which could affect \
the training phase.
In order to fix this issue, the authors decided to use \
Focal loss ~\cite{focalLoss}.

Another paper in which the authors attempt to build an aerial \
remote-controlled drone is ~\cite{drone4}.
Here, the authors experiment with several computing platforms \
and system architectures, as follows:

\begin{enumerate}
    \item \textbf{on-board embedded GPU system} the drone contains \
        a micro-processor with a high\-performance embedded GPU;
        three micro-processors were tested: Nvidia Jetson TX1,\
        Nvidia Jetson TX2 and Nvidia Xaver;
        the final image is streamed to a computer connected with a \
        Wi-Fi network
    \item \textbf{off-board GPU based station} the drone transmits \
        the raw image to a ground station equipped with a GPU, where \
        image processing occurs;
        the ground base was equipped with a GTX 1080 GPU, while the \
        drone was equipped with either a Latte Panda or an Odroid \
        microprocessor
    \item \textbf{on-board GPU-constrained system} the drone is \
        controlled by a microprocessor without a high\-performance \
        embedded GPU, but with a neural compute stick plugged in;
        the tested microprocessors were Raspberry Pi, Latte Panda \
        and Odroid
\end{enumerate}

As for the object detection algorithms, the authors experimented with \
a wide variety, including YOLO v2 \& v3, SSD, SSD MobileNet.
The best results were obtained on Xavier AGX microprocessor and on \
the GPU ground base station, as presented in the table \
~\ref{tab:drone4-results}

\begin{table}[ht]
    \caption{Results for neural networks and platforms}
    \centering
    \begin{tabular}{|c|c|c|c|c|}
        \hline\hline
        Neural Network & TX1 & TX2 & Xavier AGX & GTX 1080  \\
        \hline
        YOLO v2 & 2.9 FPS & 7 FPS & 26\-30 FPS & 28 FPS \\
        YOLO v3 & \-\-\- & 3 FPS & 16\-18 FPS & 15 FPS  \\
        YOLO v3 tiny & 9\-10 FPS & 12 FPS & 30 FPS & 30\+ FPS \\
        SSD & 8 FPS & 11\-12 FPS & 35\-48 FPS & 32 FPS \\
        \hline
    \end{tabular}
    \label{tab:drone4-results}
\end{table}


%Bibliographic research has as an objective the establishment of the references for the \
%project, within the project domain/thematic. While writing this chapter (in general the \
%whole document), the author will consider the knowledge accumulated from several \
%dedicated disciplines in the second semester, 4$^{th}$ year (Project Elaboration \
%Methodology, etc.), and other disciplines that are relevant to the project theme.
%
%Represents about 15\% of the paper.
%
%Each reference must be cited within the document text, see example below (depending \
%on the project theme, the presentation of a method/application can vary).
%
%
%This section includes citations for conferences or workshop~\cite{BellucciLZ04}, \
%journals~\cite{AntoniouSBDB07},
%and books~\cite{russell1995artificial}.
%
%In paper~\cite{AntoniouSBDB07} the authors present a detection system for moving obstacles based on stereovision and ego motion estimation.
%The method is ... {\it discus the algorithms, data structures, functionality, specific aspects related to the project theme, etc.}... Discussion: {\it pros and cons}.
%
%In chapter~\ref{ch:analysis} of~\cite{strunk}, the {\it similar-to-my-project-theme algorithm} is presented, with the following features ...
%
%
%\section{Title}
%\section{Other title}



\chapter{Analysis and Theoretical Foundation}
\label{ch:analysis}

%Together with the next chapter takes about 60\% of the whole paper
%
%The purpose of this chapter is to explain the operating principles of the implemented application.
%Here you write about your solution from a theory standpoint - i.e. you explain it and you demonstrate its theoretical properties/value, e.g.:
%\begin{itemize}
% \item used or proposed algorithms
% \item used protocols
% \item abstract models
% \item logic explanations/arguments concerning the chosen solution
% \item logic and functional structure of the application, etc.
%\end{itemize}
%
%{\color{red} YOU DO NOT write about implementation.
%
%YOU DO NOT copy/paste info on technologies from various sources and others alike, which do not pertain to your project.
%}

%\section{Title}
%\section{Other title}
\section{Algorithms}
\label{sec:analysis-algorithms}
 The project relies on several different algorithms for achieving its mission to detect and track people.
 The most notable algorithms are for image transmission, person detection, object tracking and face recognition.

\subsection{Image transmission}
\label{subsec:image-transmission}
 I needed to devise an algorithm that would allow to one source to send images to several receivers that are
 not on the same network and at considerable distances, and at the same time having a delay as small as possible.
 Since the sender and the receivers won't be on the same network, they will need to interact via a 3rd component,
 a server that runs in cloud.
 The

\subsection{Person Detection}
\label{subsec:person-detection}
Person detection consists of 2 phases: full person detection, and face detection. % maybe test just face detection
This is due to the fact that it's easier to detect and track a person than to detect and track a face.
For person detection I have chosen neural networks


 According to [1], \ %todo: insert bib ref
different low level heuristics behave differently in different environments, with some working better than others. \
In order to rank low level heuristics, 2 evaluation functions are used: \textbf{competence} and \textbf{affinity}. \
Competence describes how well a heuristic works on its own. \
Affinity is a relationship between 2 low level heuristics \
and describes how well a low level heuristic works when being applied after another low level heuristic. \
This way, \
when choosing a new low level heuristic to apply on the solution, we can use either competence or affinity to pick \
one. \
However, we also need to take into account the time passed since the low level heuristic was last used, otherwise \
some low level heuristics would never be used. \
The formula used for choosing a low level heuristic by competence is \
the following:

\( P(h) = \frac{C(h)}{\sum_{h_i \epsilon H} C(h_i)} * \Delta T_h \)

Where $P(h)$ is the probability of choosing heuristic $h$, $C(h)$ is the competence is heuristic h and $H$ is the set \
of all low level heuristics and $\Delta T_h$ is the time spent since the last time heuristic $h$ was last used. \
This way, we ensure that all heuristics are used, and not just 2 or 3 that behave the best.

The affinity formula used for choosing a heuristic is the following:

\( P(h) = \frac{Aff(h_\alpha, h)}{\sum_{h_i \epsilon H} Aff(h_\alpha , h_i)} * \Delta T_h \)

Where $P(h)$ is the probability of choosing heuristic $h$, $h_\alpha$ is the heuristic that was last applied, \
$Aff(h_\alpha , h)$ is the affinity between heuristics $h_\alpha$ and $h$ and $\Delta T_h$ is the time spent since \
heuristic $h$ was last used.

Both the competence and the affinity must be initialized in a pre-run phase and then must be updated after each \
heuristic is applied. \
In order to initialize them, we need several sequences of heuristics that have behaved well or \
simply better than other sequences. \
We shall note the set of such sequences $\Theta$. \
The formula for initializing the competence is the following:

\( C(h_A) = \sum_{H \epsilon \Theta} \sum_{h_i \epsilon H} h_A == h_i ? 1 : 0 \)

Basically, the above formula counts the number of times heuristic $h_A$ appears in the set of good heuristic \
sequences. \
The formula for initializing affinity is the following:

\(  Aff(h_A, h_B) = \sum_{H \epislon \Theta} \sum_{i=1}^{len(H) - 2} \sum_{j=i+1}^{len(H) - 1} h_A == h_i \wedge \
h_B == h_j ? \frac{1}{j - i} : 0 \) %todo: fix &&

Thus, the affinity between 2 heuristics is maximum when they are placed one after another (are adjacent) and decrease \
as the distance between them grows.

In order to choose between competence and affinity when deciding which heuristics to use next, we can define a \
constant $\alpha,  0 < \alpha < 1$, and then generate a random number between 0 and 1. \
If the number is smaller than $\alpha$, we will use competence, otherwise affinity. %todo: validate statement with alpha


\subsection{Simulated Annealing Meta Heuristic}
\label{subsec:analysis-simulated-annealing}
The term \textit{Simulated Annealing} originates from metallurgy, where \textit{annealing} refers to a technique that \
consists of heating a material above its melting point and then gradually cooling it in order to reduce its defects \
and increase the size of its crystals.
In Computer Science, \textit{Simulated Annealing} is a simulation technique used to detect global optimum in an \
environment with many local optimums.

According to \cite{springer1}, \ %todo: fix citation and complete this


\subsection{Proposed Hyper Heuristic}
\label{subsec:analysis-proposed-hh}
The proposed algorithm is a combination of the choice function hyper heuristic and the simulated annealing \
meta-heuristic. \
The algorithm steps can be found below: %todo: add label to algorithm

\noindent\rule{\textwidth}{1pt}
\begin{enumerate}[itemsep=1pt, parsep=1pt, topsep=1pt]
    \item Generate an initial solution of the optimization problem, $sol_{domain}$, randomly
    \item Generate a set of low level heuristics sequences randomly
    \item Apply each sequence of the low level heuristic on the same $sol_{domain}$
    \item \begin{enumerate}[itemsep=1pt, parsep=1pt, topsep=1pt]
              \item The best resulting solution will be $sol_{domain\_opt}$
              \item Save the best n sequences
    \end{enumerate}
    \item Using the best n sequences, initialize the competence and affinity using the equations presented above. %todo: link competence and affinity
    \item Initialize temperature T
    \item While the temperature T is above a certain threshold:
    \item \begin{enumerate}[itemsep=1pt, parsep=1pt, topsep=1pt]
              \item Pick a low level heuristic h using either competence or affinity
              \item Apply h on $sol_{domain}$; the resulting solution will be $sol_{domain1}$
              \item If $sol_{domain1}$ is better than $sol_{domain}$:
              \begin{enumerate}[itemsep=1pt, parsep=1pt, topsep=1pt]
                  \item Update the competence and affinity of heuristic h
                  \item $sol_{domain}$ will take the value of $sol_{domain1}$
                  \item if $sol_{domain1}$ is better than $sol_{domain\_opt}$, than $sol_{domain\_opt}$ will take the value of $sol_{domain1}$
              \end{enumerate}
              \item If $sol_{domain1}$ is not better than $sol_{domain}$
              \begin{enumerate}[itemsep=1pt, parsep=1pt, topsep=1pt]
                  \item $sol_{domain}$ will take the value of $sol_{domain1}$ with the probability $P = e^{\frac{-\Delta E}{T}}$
              \end{enumerate}
              \item Update the temperature $T = T - \Delta T$
    \end{enumerate}
\end{enumerate}
\noindent\rule{\textwidth}{1pt}

As can be seen from the algorithm, the choice function hyper-heuristic is used to select the next heuristic to be \
applied on the solution, while simulated annealing is used to sometimes pick a worse solution in order to escape \
local minimums. \
The main idea of this algorithm is not to obtain the best solution possible, but a good enough \
solution in a reasonable time.


%todo: complete here
\subsection{Abstract Models}
\label{subsec:analysis-models}
The search space is represented by meals. \
These are characterized by multiple properties:
\begin{itemize}
    \item \textbf{Name} Name of the meal
    \item \textbf{Type} Must be one of the following: \textit{Breakfast, First Snack, Lunch, Second Snack, Dinner}
    \item \textbf{Distributor} A distribution company with an address from which meals will be delivered
    \item \textbf{Nutritional components}
    \item \textbf{Food Items} The raw ingredients the meal is made from.
    \item \textbf{Reliability} is a combination of aspect, taste and smell
    \item \textbf{Price}
    \item \textbf{Time} Time it takes to cook and deliver the meal
\end{itemize}
A menu (recommendation) is made of 5 meals of different types: breakfast, first snack, lunch, second snack and dinner.




\subsection{Low Level Heuristics}
\label{subsec:analysis-llh}
The proposed hyper heuristic uses 9 different low level heuristics, most of which were taken from genetic algorithms:
\begin{itemize}
    \item \textbf{Optimum Single Point Mutation} for each type of meal, it retrieves a meal from the database of the \
same type. If the new meal is better than the current meal, than it replaces the current meal and then the heuristic \
stops. If no new meal is better than its current equivalent, the current menu remains unchanged.
    \item \textbf{Optimum Single Point Crossover} uses the optimum domain solution as a reference solution. For each \
type of meal, it checks if the optimum solution meal is better than the current meal. If so, it replaces the current \
meal with the optimum solution meal and then it stops. If no meal from the optimum solution is better than its current \
equivalent, than the current meal remains unchanged.
    \item \textbf{Optimum Multiple Point Mutation} is similar to the optimum single point mutation. the only \
difference is that it doesn't stop once one meal has been replaced. If all new meals retrieved from the database are \
better than their current equivalents, than all current meals will be replaced.
    \item \textbf{Optimum Multiple Point Crossover} is similar to the optimum single point crossover, the only \
difference being the same difference as between single point and multiple point mutation
    \item \textbf{Random Single Point Mutation} replaces one random meal with a new one from the database. The new \
meal doesn't have to be better than the current one.
    \item \textbf{Random Single Point Crossover} replaces one random meal with a meal of the same type from the optimum \
solution.
    \item \textbf{Random Multiple Point Mutation} replaces all current meals with new one from the database
    \item \textbf{Random Multiple Point Crossover} replaces the current menu with the optimum menu
    \item \textbf{Memory Based Mutation} each time a meal is replaced with another meal, the difference in score \
between the before and after menu is saved in a memory. This heuristic uses that memory to try and replace all meals \
from the current solution with meals from memory that have yielded the largest score increase
\end{itemize}

%todo: complete this
\section{User Profile}
\label{sec:analysis-user-profile}
The user profile consists of information about allergies to food items and nutrient intake values.

%todo: retun here
\section{Fitness Function}
\label{sec:analysis-fitness}
This Fitness function is ued to evaluate the quality of a menu or of a meal from a menu. \
It is inverse proportional with te quality of a meal. \
Thus, the better the meal/menu, the lower the fitness. \
When computing the fitness of a menu, several factors must be taken into consideration:
\begin{itemize}
    \item \textbf{Nutritional values} for each meal type (breakfast, lunch, dinner, 2 snacks)
    \item \textbf{Price} the price of the meal should not be much higher than the consumer desired price
    \item \textbf{Time}
    \item \textbf{Reliability}

\end{itemize}

\section{Structure of Application}
\label{sec:analysis-structure}
The application consists of 3 main components:
\begin{itemize}
    \item \textbf{Data Access Component} Is used to retrieve users, users profile and meals from database. This is a \
            component used by multiple algorithms used to generate menus: choice function, cuckoo search.
    \item \textbf{Main Algorithm} This is the hyper-heuristic algorithm used to generate a menu
    \item \textbf{Web component} that all users (elders, nutritionists, distributors) will interact with.
\end{itemize}


\chapter{Detailed Design and Implementation}
\label{ch:implementation}

% todo: detaliere fiecare chema

% results: rezultate intermediare, rezultate finale (hardware timp de raspuns, timp duratie transmitere, timp procesare imagine, tabele acuracy, procentaj detectie fete, procentaj recunoastere fete)
% pe cd: cod + documentatie
% gdpr compliance

\section{Introduction}
\label{sec:implementation-introduction}
This project contains 4 different components that are deployed in different \
places:
\begin{enumerate}
    \item the robot application (deployed and running on the actual robot)
    \item the proxy server that acts as an intermediary between the user \
            controlling the robot and the robot; \
            it runs in a kubernetes \
            cluster in the cloud (GKE, more precisely)
    \item an angular web application that is used to control the robot; \
            it is \
            deployed in the same cloud as the proxy server, but runs in the \
            user's browser
    \item the algorithm used to split an image into several UDP-ready packets \
            and to reconstruct the image from said packets; \
            the algorithm is published as a public package that is imported by \
            both the robot application and the web application
\end{enumerate}

Each component's detailed design and implementation will be detailed below.

%Together with the previous chapter takes about 60\% of the paper.
%
%The purpose of this chapter is to document the developed application such a way that it can be maintained and \
%developed later. \
%A reader should be able (from what you have written here) to identify the main functions of the application.
%
%The chapter should contain (but not limited to):
%\begin{itemize}
%    \item a general application sketch/scheme,
%    \item a description of every component implemented, at module level,
%    \item class diagrams, important classes and methods from key classes.
%\end{itemize}


\chapter{Testing and Validation}

About 5\% of the paper
\section{Title}
\section{Other title}

\chapter{User's manual}

\section{Hardware}
In order to run the application, an administrator requires at leat a \
physical or virtual machine with a x64 processor, 2 GB RAM and 50GB hard disk.\
Ideally, the machine should be running a Linux OS, but Windows is also supported.

\section{Software Dependencies}
In the following section, we will present the installation steps on Debian-based OS \
(Debian, Ubuntu, Mint).

There are 3 main dependencies: Apache Tomcat(the environment used to run the \
application server), MySQL(used to host the application database) and NginX (web \
server which will be used as a reverse proxy and where we will setup the SSL encryption).\
However, before installing any software the administrator must update the package \
repositories by running the following command in a terminal:

%\begin{verbatim}
%sudo apt update
%\end{verbatim}
%
%\subsection{MySQL}
%In order to install MySQL server, you need to run the following commands:
%\begin{verbatim}
%apt install mysql-server
%/etc/init.d/mysql start
%\end{verbatim}
%
%Then, you need to add a custom admin user:
%\begin{verbatim}
%mysql -u root -p
%mysql>CREATE USER 'user'@'%' IDENTIFIED BY 'password';
%mysql>GRANT ALL PRIVILEGES ON *.* TO 'user'@'%' WITH GRANT OPTION;
%mysql>FLUSH PRIVILEGES;
%mysql>quit;
%\end{verbatim}
%Then, you need to edit the \textit{my.cnf} file to allow remote connections to the \
%mysql server. Usually it is located at \textit{/etc/mysql/my.cnf}, but this may change \
%depending on the OS. You need to change the \textit{bind-address} option as below:
%\begin{verbatim}
%bind-address = 0.0.0.0
%\end{verbatim}
%
%Then restart the mysql service with the command \textit{sudo /etc/init.d/mysql restart}.
%
%
%\subsection{Apache Tomcat}
%
%In order to install Apache Tomcat, a user first needs to install Java on the system. \
%This can be done by running the following command (This will install both the runtime \
%environment and the development kit.):
%
%\begin{verbatim}
%sudo apt install default-jdk
%\end{verbatim}
%
%In order to validate the installation, run the following commands:
%\begin{verbatim}
%java -version
%javac -version
%\end{verbatim}
%If no error is presented, the the java environment was installed successfully.
%
%Next, you need to add a \textit{tomcat} user and user group. The home directory will \
%be \textit{/opt/tomcat}, and the shell \textit{/bin/false}, so that no user can login \
%as tomcat.
%
%\begin{verbatim}
%sudo groupadd tomcat
%sudo mkdir -p /opt/tomcat
%sudo useradd -s /bin/false -g tomcat -d /opt/tomcat tomcat
%\end{verbatim}
%
%Next, you need to install additional dependencies:
%\begin{verbatim}
%sudo apt install curl
%\end{verbatim}
%Then, download the tomcat sources and extract them:
%\begin{verbatim}
%cd /tmp
%curl -O apache.mirrors.ionfish.org/tomcat/tomcat-9/v9.0.10/src/apache-tomcat-9.0.10-src.tar.gz
%tar xzvf apache-tomcat-9.0.10-src.tar.gz -C /opt/tomcat --strip-components=1
%\end{verbatim}
%
%Next, you need to update user permissions:
%\begin{verbatim}
%cd /opt/tomcat
%sudo chgrp -R tomcat /opt/tomcat
%sudo chmod -R g+r conf
%sudo chmod g+x conf
%sudo chown -R tomcat webapps/ work/ temp/ logs/
%\end{verbatim}
%
%Next, you need to create the service file. For that, you need the \
%\textit{JAVA\_HOME}. In order to obtain it, run the command: \
%\textit{sudo update-java-alternatives -l}. The output should be \
%something similar to below:
%\begin{verbatim}
%java-1.8.0-openjdk-amd64       1081       /usr/lib/jvm/java-1.8.0-openjdk-amd64
%\end{verbatim}
%
%The \textit{JAVA\_HOME} can be constructed by appending \textit{/jre} \
%to the last column. Now you can create the file \
%\textit{/etc/systemd/system/tomcat.service} and put the following \
%contents in it:
%
%\begin{verbatim}
%[Unit]
%Description=Apache Tomcat Web Application Container
%After=network.target
%
%[Service]
%Type=forking
%
%Environment=JAVA_HOME=/usr/lib/jvm/java-1.8.0-openjdk-amd64/jre
%Environment=CATALINA_PID=/opt/tomcat/temp/tomcat.pid
%Environment=CATALINA_HOME=/opt/tomcat
%Environment=CATALINA_BASE=/opt/tomcat
%Environment='CATALINA_OPTS=-Xms512M -Xmx1024M -server -XX:+UseParallelGC'
%Environment='JAVA_OPTS=-Djava.awt.headless=true -Djava.security.egd=file:/dev/./urandom'
%
%ExecStart=/opt/tomcat/bin/startup.sh
%ExecStop=/opt/tomcat/bin/shutdown.sh
%
%User=tomcat
%Group=tomcat
%UMask=0007
%RestartSec=10
%Restart=always
%
%[Install]
%WantedBy=multi-user.target
%\end{verbatim}
%
%Next, you need to reload the systemd daemon in order for it to discover \
%the new tomcat service, and then you can start the service.
%\begin{verbatim}
%sudo systemctl daemon-reload
%sudo systemctl start tomcat
%\end{verbatim}
%In the installation description section your should detail the hardware \
%and software resources needed for installing and running the application, \
%and a step by step description of how your application can be \
%deployed/installed. An administrator should be able to perform the \
%installation/deployment based on your instructions.
%
%In the user manual section you describe how to use the application from \
%the point of view of a user with no inside technical information; this \
%should be done with screen shots and a stepwize explanation of the interaction. \
%Based on user's manual, a person should be able to use your product.

\chapter{Conclusions}

About. 5\% of the whole

Here your write:
\begin{itemize}
\item a summary of your contributions/achievements,
\item a critical analysis of the achieved results,
\item a description of the possibilities of improving/further development.
\end{itemize}
\section{Title}
\section{Other title}


%\addcontentsline {toc}{chapter}{Bibliography}
\bibliographystyle{IEEEtran} 
\bibliography{thesis}%same file name as for .bib


\appendix
\chapter{Relevant code}
\label{ch:relevant-code}

\section{Engine and Motor Control}
\label{sec:annex-engine-motor-control}
\begin{verbatim}
from adafruit_motor.motor import DCMotor

class Motor:

    def __init__(self, adafruit_motor: DCMotor, forward=1):
        self.motor = adafruit_motor
        self.forward_direction = forward
        self.throttle = 0

    def move(self, throttle):
        self.motor.throttle = throttle * self.forward_direction


class Engine:

    def __init__(self, motor1: Motor, motor2: Motor, motor3: Motor, motor4: Motor):
        self.throttle = 0
        self.top_left_motor = motor1
        self.bottom_left_motor = motor2
        self.bottom_right_motor = motor3
        self.top_right_motor = motor4
        self.all_motors = [self.top_left_motor, self.bottom_left_motor, self.bottom_right_motor, self.top_right_motor]

    def forward(self, throttle):
        print('Forward', throttle)
        self.throttle = throttle
        for motor in self.all_motors:
            motor.move(self.throttle)

    def reverse(self, throttle):
        print('Reverse', throttle)
        self.throttle = -1 * throttle
        for motor in self.all_motors:
            motor.move(self.throttle)

    def steer(self, direction):
        print('Steer', direction)
        """
        :param direction:
            -1 -> left
            1 -> right
        :return:
        """
        if direction == 0:
            self.forward(self.throttle)
        elif direction < 0:
            self._steer_left(direction * -1)
        else:
            self._steer_right(direction)

    def _steer_right(self, intensity):
        self.bottom_right_motor.move(self.throttle * intensity)
        self.top_right_motor.move(self.throttle * intensity)

    def _steer_left(self, intensity):
        self.top_left_motor.move(self.throttle * intensity)
        self.bottom_left_motor.move(self.throttle * intensity)

    def stop(self):
        print('Stop')
        for motor in self.all_motors:
            motor.move(0)

\end{verbatim}


\end{document}

\begin{document}



\end{document}
\begin{document}



\end{document}
\begin{document}



\end{document}

